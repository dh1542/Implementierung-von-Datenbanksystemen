\documentclass[10pt,a4paper]{article}
\usepackage[utf8]{inputenc}
\usepackage[T1]{fontenc}
\usepackage{amsmath}
\usepackage{amsfonts}
\usepackage{amssymb}
\usepackage{graphicx}
\usepackage[hidelinks]{hyperref}
\usepackage{listings}
\usepackage[left=2.00cm, right=2.00cm, top=0.00cm, bottom=2.00cm]{geometry}
\begin{document}
\title{IDB Gitlab -- Ein Einstieg}
\date{}
\maketitle
\section{Kurzeinführung}
Hier gibts eine kurze\footnote{\url{https://rogerdudler.github.io/git-guide/}}
und etwas ausführlichere Anleitung\footnote{\url{https://git-scm.com/book/en/v2}}
für git.
\section{Einrichtung gitlab.cs.fau.de}
Für den Zugriff auf ein Repo gibt es generell zwei Möglichkeiten:
\begin{enumerate}
\item Nutzername und Passwort
\item (SSH-)Schlüsselpaar
\end{enumerate}
Hierfür benötigt es jeweils ein paar Vorbedingungen:
\begin{enumerate}
\item Hierfür muss man auf GitLab sein Passwort festlegen\footnote{\url{https://gitlab.cs.fau.de/profile/password/edit}}.
\item Hierfür muss ein Schlüsselpaar angelegt werden\footnote{\href{https://linux.die.net/man/1/ssh-keygen}{ssh-keygen(1)}} und der öffentliche Schlüssel in GitLab eingetragen werden\footnote{\url{https://gitlab.cs.fau.de/profile/keys}}.
\end{enumerate}
\section{Einrichtung des Repositories}
Hierfür benötigt es ein paar Links, welche sich je nach Zugriffsart unterscheiden. Bei Zugriff per Schlüsselpaar sind das die mit \lstinline|git@gitlab.cs.fau.de:idbprog/| beginnenden, bei der Authentifizierung mit Nutzername und Passwort die mit \lstinline|https://gitlab.cs.fau.de/idbprog/| beginnenden.
Die genauen Zeichenketten für die \$VORLAGE und Ihr persönliches Repository \$MYREPO finden Sie im jeweiligen Repository beim klicken auf den \textit{CLONE}-Knopf
\begin{lstlisting}
git clone $MYREPO idbprog
cd idbprog
git remote add vorlage $VORLAGE
git pull vorlage master
git push
\end{lstlisting}

Beim Pushen kann es zu Fehlern kommen, wenn Ihr Repository (\$MYREPO) Änderungen besitzt, die nicht lokal vorhanden sind.
Dies kann man auf verschiedene Arten lösen:
\begin{itemize}
\item Überschreiben: \lstinline|git push --force|\\
Anmerkung: Hierfür muss der Masterbranch auf \url{gitlab.cs.fau.de} in den Einstellungen aus den Protecteten Branches entfernt werden.
\item Kombinieren der beiden Vergangenheiten durch anhängen der eigenen Vergangenheit an die Remote Vergangenheit: 
\begin{lstlisting}
git pull --rebase
git push
\end{lstlisting}
\item Kombinieren der beiden Vergangenheiten durch einen Merge:
\begin{lstlisting}
git fetch
git merge FETCH_HEAD
git push
\end{lstlisting}
\end{itemize}
\section{Git-Cheatsheet}
Es gibt mehrere Cheat Sheets online.\footnote{\url{https://about.gitlab.com/images/press/git-cheat-sheet.pdf}}\footnote{\url{https://training.github.com/downloads/github-git-cheat-sheet.pdf}}
\end{document}