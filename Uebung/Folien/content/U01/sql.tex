\section{SQL}
\sectionpage

\begin{frame}{Gegeben ist der folgende Teil eines einfachen Datenbankschemas:}
\texttt{songs(\underline{title, artist}, length, album[albums])}
\newline
\texttt{producers(\underline{name}, date\_of\_birth)}
\newline
\texttt{albums(\underline{title}, producer[producers], release\_year, genre)}
\end{frame}

\begin{frame}{Erstellen Sie eine SQL-Anfrage, die alle Titel von Ray Charles' Songs liefert, die höchstens 180 Sekunden dauern!}
\only<+->{
\texttt{songs(\underline{title, artist}, length, album[albums])}
\newline
\texttt{producers(\underline{name}, date\_of\_birth)}
\newline
\texttt{albums(\underline{title}, producer[producers], release\_year, genre)}
}

\vspace{2em}
\only<+->{
SELECT title
\newline
FROM   songs
\newline
WHERE  artist = 'Ray Charles' and length <= 180;}
\end{frame}

\begin{frame}{Erstellen Sie eine SQL-Anfrage, die jeden Songtitel zusammen mit dem Geburtsdatum des Produzenten des Albums ausgibt, auf dem der Song erschienen ist.}
\only<+->{
\texttt{songs(\underline{title, artist}, length, album[albums])}
\newline
\texttt{producers(\underline{name}, date\_of\_birth)}
\newline
\texttt{albums(\underline{title}, producer[producers], release\_year, genre)}
}

\vspace{2em}
\only<+->{
SELECT s.title, p.date\_of\_birth

FROM   songs s, producers p, albums a

WHERE  s.album = a.title AND a.producer = p.name;
}
\end{frame}

\begin{frame}{Erstellen Sie eine SQL-Anfrage, die eine aufsteigend nach Künstlername sortierte Liste der Künstler und der Anzahl an Songs, die der jeweilige Künstler in der Datenbank hat, zurückgibt. Berücksichtigen Sie dabei nur Songs, die auf einem Album erschienen sind und berücksichtigen Sie nur Künstler, zu denen es mindestens 5 solcher Songs gibt.}
\only<+->{
\texttt{songs(\underline{title, artist}, length, album[albums])}
\newline
\texttt{producers(\underline{name}, date\_of\_birth)}
\newline
\texttt{albums(\underline{title}, producer[producers], release\_year, genre)}
}

\vspace{2em}
\only<+->{
SELECT   artist, count(title) AS anzahl
\newline
FROM     songs
\newline
WHERE    album IS NOT NULL\\
GROUP BY artist\\
HAVING   count(title) >= 5\\
ORDER BY artist ASC;
}
\end{frame}

\begin{frame}{Erstellen Sie eine SQL-Anfrage, die die 5 Alben mit der durchschnittlich längsten Song-Dauer in einem Ranking ausgibt, wobei das Album mit den durchschnittlich längsten Songs Rang 1 belegt.}

\texttt{songs(\underline{title, artist}, length, album[albums])}
\newline
\texttt{producers(\underline{name}, date\_of\_birth)}
\newline
\texttt{albums(\underline{title}, producer[producers], release\_year, genre)}

\end{frame}

\begin{frame}{Mögliche Lösung:}

WITH avg\_songlength AS
\newline
(
\newline
	SELECT   album, AVG(length) AS avg\_song\_length
	\newline
	FROM     songs
	\newline
	GROUP BY album
 \newline
)
\newline
SELECT   COUNT(*) AS rang, a1.album, a1.avg\_song\_length
\newline
FROM     avg\_songlength a1, avg\_songlength a2
\newline
WHERE    a1.avg\_song\_length <= a2.avg\_song\_length
\newline
GROUP BY a1.album, a1.avg\_song\_length
\newline
HAVING   COUNT(*) <= 5
\newline
ORDER BY rang ASC;

\end{frame}

\begin{frame}{Erstellen Sie eine SQL-Anfrage, die alle Songs ausgibt, die länger sind als der Durchschnitt aller Songs. Erweitern Sie die Anfrage dann so, dass sie alle Songs ausgibt, die länger sind als der Durchschnitt aller Songs des jeweiligen Albums.}
\only<+->{
\texttt{songs(\underline{title, artist}, length, album[albums])}
\newline
\texttt{producers(\underline{name}, date\_of\_birth)}
\newline
\texttt{albums(\underline{title}, producer[producers], release\_year, genre)}
}

\vspace{2em}
\only<+->{
\begin{minipage}{0.4\textwidth}
SELECT *
\newline
FROM   songs s1
\newline
WHERE  s1.length > 
\newline
(
\newline
SELECT avg (s2.length)\\
FROM songs s2\\
);
\end{minipage}}
\only<+->{
	\begin{minipage}{0.4\textwidth}
		SELECT *
		\newline
		FROM   songs s1
		\newline
		WHERE  s1.length > 
		\newline
		(
		\newline
		SELECT avg (s2.length)\\
		FROM songs s2\\
		WHERE s1.album = s2.album\\
		);
	\end{minipage}
}
\end{frame}
