\section{LINQ}
\begin{frame}{Was ist LINQ?}
\begin{itemize}
	\item Möglichkeit zur Abfrage von Daten
\begin{itemize}
\item deklarativ
\item SQL-ähnliche Syntax
\item aus
\begin{itemize}
\item  relationalen Datenbanken
\item Arrays, Listen, \ldots
\item XML-Strukturen
\item \ldots (beliebig erweiterbar)
\end{itemize}
\end{itemize}
\item  Integriert in das .NET Klassensystem und die Programmiersprache
\begin{itemize}
\item Syntaxkontrolle durch den Compiler
\item Strenge Typisierung
\item IDE-Unterstützung (Codevervollständigung, …)
\end{itemize}
\end{itemize}
\end{frame}

\begin{frame}{LINQ to Entities}
\begin{itemize}
\item Object-relational mapping\\
	Normale .NET-Klasse zur Datenrepräsentation
\item Abfragen über LINQ-Syntax\\
	Übersetzung in SQL
\item Transaktionsunterstützung
\item Möglichkeit zur Erzeugung der Datenbank aus Klassendefinitionen und umgekehrt
\end{itemize}
\end{frame}

\lstdefinestyle{sharpc}{
language=[Sharp]C,
tabsize=2,
morekeywords={
	from,
	where,
	orderby,
	select},
rulecolor=\color{blue!80!black}}

\begin{frame}[fragile]{Ein erstes Beispiel (C\#)}
\begin{lstlisting}[style=sharpc]
sting[] namen = {"Tina", "Tom", "Hans", "Susanne", "Gabriel", "Tim", "Michael", "Daniel"};

IEnumerable<string> query = from s in namen
	where s.Length == 3
	orderby s
	select s.ToUpper();

foreach (string item in query)
	Console.WriteLine(item);
\end{lstlisting}
\end{frame}

\begin{frame}[fragile]{Andere Syntax}
\begin{lstlisting}[style=sharpc]
sting[] namen = {"Tina", "Tom", "Hans", "Susanne", "Gabriel", "Tim", "Michael", "Daniel"};

IEnumerable<string> query = from s in namen
	.Where(s => s.Length == 3)
	.OrderBy(s => s)
	.Select(s => s.ToUpper());

foreach (string item in query)
	Console.WriteLine(item);
\end{lstlisting}
\end{frame}


\begin{frame}[fragile]{Mit strukturierten Werten}
\begin{lstlisting}[style=sharpc]
Person[] personen = {
	new Person {Name = "Thomas", Alter = 22},
	new Person {Name = "Susi", Alter = 17},
	new Person {Name = "Tina", Alter = 23},
}

var query = from p in personen
	select new {p.name, Volljährigkeit = p.Alter >= 18 };

foreach (var item in query)
	Console.WriteLine("{0} ist {1}", item.Name, item.Volljähring ? "volljähring" : "nicht volljähring");
\end{lstlisting}
\end{frame}
