\title[Prolog -- UeIDB]{Prolog -- Übungen zu \glqq Implementierung von Datenbanksystemen\grqq}
\author{Demian E. Vöhringer}

\begin{document}
\maketitle

\quoteframe{Man hilft den Menschen nicht, wenn man für sie tut, was sie selbst tun können.}{Abraham Lincoln}{12.02.1809 -- 15.04.1865}

\fullpictureframe[Sinn der Übung]{Pictures/IDB-00-Prolog/sinn_uebung}

\begin{frame}{Präsenz im WS 2021/2022}
\Large
\begin{itemize}
	\item OP-Maskenpflicht
	\item Kontaktdatenerfassung mit Selbstauskunft\\
	QR-Code oder Analog
	\item Stichprobenhafte 3G Kontrolle bei 10\% der Anwesenden
	\item Konsequenzen bei Teilnahme ohne 3G\\
	Tutor muss Personalien aufnehmen und des Gebäudes verweisen.\\
	Unileitung kann Verfahren wg. Ordnungswidrigkeit \alert{(Bußgeld 250€)} einleiten oder beschränktes \alert{Hausverbot} erteilen.
\end{itemize}
\end{frame}

\begin{frame}{StudOn}
\Large
Sind alle Anwesenden in StudOn angemeldet?
\begin{itemize}
\item Zugriff auf Unterlagen
\item Möglichkeiten zur freiwilligen Abgabe von Übungsaufgaben
\item Forum für Diskussionen und Ankündigungen
\item Möglichkeit zur Benachrichtigung aller Teilnehmer per E-Mail
\item Gleichmäßige Verteilung auf Übungsgruppen
\end{itemize}
\end{frame}

\begin{frame}{Das Übungsblatt}
\Large
\begin{itemize}
\item Präsenzaufgaben
\begin{itemize}
\item Eigenständig vorbereiten
\item Besprechung in der Tafelübung
\item Musterlösung wird veröffentlicht
\end{itemize}
\item $\ast$ Vertiefungsaufgaben
\begin{itemize}
\item Vertiefen oder erweitern den Stoff
\item Abgabe auf StudOn
\item Korrektur durch Tutoren
\item Keine Musterlösung
\item \alert{Neu:} $\ast\ast$ Einige sind per Selbsttest abzugeben
\begin{itemize}
	\item Werden nicht durch Tutoren korrigiert.
	\item Können öfter durchgeführt werden.
\end{itemize}
\end{itemize}
\item $\ast$ Programmieraufgaben
\begin{itemize}
\item Praxiserfahrung
\item Abgabe auf GitLab
\item Korrektur durch Testcases und Tutor
\end{itemize}
\end{itemize}
\end{frame}


\begin{frame}{Übungswoche}
\large
\begin{itemize}
\item Übungsblätter mit Präsenz- und Vertiefungsaufgaben auf StudOn\\
{\small Nach der letzten Übung der Vorwoche}
\item Eigenständiges Vorbereiten der Präsenzaufgaben
\item Besprechung der Aufgaben in den Übungen\\
{\small In der Übungswoche}
\begin{itemize}
\item Kein Monolog der Tutoren
\item aktive Mitarbeit und Diskussion
\item Fragen stellen
\item Lösungsalternativen
\end{itemize}
\item Musterlösung der Präsenzaufgaben auf StudOn\\
{\small Nach der letzten Übung der Übungswoche}
\item Abgabe der Zusatzaufgaben bis zur Intensivierungsübung der Folgewoche\\
\alert{Donnerstag 12 Uhr}
\end{itemize}
\end{frame}


\begin{frame}{Unklarheiten}
\Large
\begin{minipage}{0.7\textwidth}
Wen bzw. wo Fragen?
\begin{enumerate}
\item Kommilitonen
\item Forum in StudOn
\item Vorlesungsbesprechung
\item Intensivierungsübung
\item Sprechstunde Demian E. Vöhringer\\
\href{https://cs6.tf.fau.de/dev}{cs6.tf.fau.de/dev}
\item Sprechstunde Prof. Viktor Leis\\
\href{https://cs6.tf.fau.de/leis}{cs6.tf.fau.de/leis}
\end{enumerate}
\end{minipage}
\begin{minipage}{0.25\textwidth}
\begin{tikzpicture}
\node[alice, monitor, minimum size=1.5cm, mirrored] (a) at (3,3) {};
\node[cloud callout, draw, callout absolute pointer=(a.70)] (sql) at(4,5.5) {???};
\end{tikzpicture}
\end{minipage}
\end{frame}

\begin{frame}{Intensivierungsübung}

\begin{tikzpicture}
\node[alice, monitor, minimum size=1.5cm] (a) at (0,0) {};
\node[rectangle callout, draw, fill=background, callout absolute pointer=(a.100), align=center, rounded corners=10] (Frage1) at(-3.5,5) {Fragen zum\\Vorlesungsstoff?};
\node[rectangle callout, draw, fill=background, callout absolute pointer=(a.120), align=center, rounded corners=10] (Frage2) at(-3.5,2.5) {Offene Fragen\\nach der\\Besprechung\\des\\Übungsblatts?};
\node[rectangle callout, draw, fill=background, callout absolute pointer=(a.140), align=center, rounded corners=10] (Frage3) at(-3.5,0) {Weitergehnde\\Fragen zum\\Übungsstoff?};

\node[ellipse callout, draw, fill=background, callout absolute pointer=(a.60), align=center] (Antwort) at(3.5,2.5) {Intensivierungsübung\\Do 12:15\\ Bitte pünktlich zu Beginn \\ besuchen!};
\end{tikzpicture}
\end{frame}

%\begin{frame}{Arbeitsaufwand \& Zeitmanagement}
%Durchschnittliche Studierende sollten in zwei Stunden pro Woche die Aufgaben lösen und vertiefen können.
%
%{\small(Zzgl. Nachbereitung der Vorlesung: mind. 1h/w)}
%
%\end{frame}

\section{Prüfung} \sectionpage
\Large
\begin{frame}{Prüfung}
\begin{itemize}
\item Schriftliche Prüfung
\begin{itemize}
\item Üblicherweise Freitag in der ersten Prüfungswoche
\item 90 Min
\end{itemize}
\item Prüfungsstoff
\begin{itemize}
\item Vorlesung \alert{und} Übung
\item Weder Übungsstoff alleine noch Vorlesungsstoff alleine reichen zum Bestehen!
\end{itemize}
\item Prüfungsvorbereitung:
\begin{itemize}
\item Nachbereiten der Vorlesung
\item Selbstständiges bearbeiten der Aufgaben
\item Übungsteilnahme
\end{itemize}
\item Anmeldung auf \href{https://www.campus.fau.de/}{meinCampus} nicht vergessen!
\end{itemize}
\end{frame}

\begin{frame}[c]
\begin{center}
\begin{tikzpicture}
\node[bob, minimum size=3cm, mirrored, monitor] (b) {};
\node[ellipse callout, draw, callout absolute pointer=(b.mouth)] (Fragen) at(-3.5,2.5) {Fragen?};
\end{tikzpicture}
\end{center}
\end{frame}
\end{document}

