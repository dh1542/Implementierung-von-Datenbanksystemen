\item Erläutern Sie den Ablauf eines Nested-Loop Joins.

\begin{solution}
Siehe VL-Folie~\NestedLoopJoin
\end{solution}

\cprotEnv
\begin{note}
\begin{lstlisting}
R,S: Relationen
for s in S:
  if(Vorbedingung auf S erfüllt):
    for r in R:
      if(Vorbedingung auf R erfüllt)
        if(Verbundbedingung erfüllt):
          Übernimm kombination in Ergebnis
\end{lstlisting}
Komplexität: $O(pages(R)\times pages(S))$
\end{note}

\item Erläutern Sie den Ablauf eines Hash Joins nach dem Classic-Hashing-Verfahren.

\begin{solution}
Siehe VL-Folie~\HashJoin
\end{solution}

\cprotEnv
\begin{note}
\begin{lstlisting}
R: Kleinere Relation
S: größere Reation
Fragmentiere R, sodass Fragmente R_i, i=1, ..., p 
	in Hauptspeicher passen
forall R_i:
  h: empty Hashtable
  forall r in R_i:// Befüllen
    insert r into h
  forall s in S: // Probing
    if(Verbundpartner in h):
      verbinde r mit s;
\end{lstlisting}
Komplexität: $O(p\times pages(S))$\\
Idealfall: p=1; R passt komplett in den Hauptspeicher.
\end{note}
