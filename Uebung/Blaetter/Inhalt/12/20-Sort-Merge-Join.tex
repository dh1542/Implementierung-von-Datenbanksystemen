\section{Sort-Merge Join}
\label{sec:sort_merge}

\begin{enumerate}[a)]

\item \label{item:sortierung} Entwerfen Sie einen einfachen Algorithmus, der eine Relation sortiert.
	Denken Sie daran, dass der Inhalt der Relation im Allgemeinen nicht komplett in den Puffer passt.
	Geben Sie auch an, wie viele Blockzugriffe auf den Hintergrundspeicher für die Sortierung benötigt werden.

	\begin{solution}
		Die Grundidee besteht darin, Teillisten zu sortieren und diese anschließend zusammenzuführen (Merge).
		Seien \(R\) die zu sortierende Relation mit \(B(R)\) Blöcken und \(M\) die Anzahl verfügbarer Kacheln im Puffer.
		Dann kann \(R\) mit folgendem Algorithmus sortiert werden:

		\begin{enumerate}[1.]
		\item Lies die ersten \(M\) Blöcke von \(R\) in den Puffer.
		\item Sortier diese Blöcke mit einem Hauptspeicher-Sortieralgorithmus.
		\item Schreib die sortierten Blöcke zurück auf den Hintergrundspeicher.
		\item Wiederhol die Schritte 1 bis 3 für alle \(B(R)\) Blöcke.
		\item Reservier eine Kachel für das Ergebnis.
		\item Lies den ersten Block jeder sortierten Teilliste in den Puffer.
		\item Lies das kleinste Elemente der Blöcke im Puffer, füg es in den Ausgabeblock ein.
		\item Schreib den Ausgabeblock auf den Hintergrundspeicher, sobald er voll ist.
		\item Wenn ein Eingabeblock komplett verarbeitet wurde, lade den nächsten Block der Teilliste in den Puffer.
		\end{enumerate}

		Zahl der Blockzugriffe: \(4 B(R)\)
		Falls die sortierten Tupel gleich weiterverarbeitet werden, spart das das Schreiben des Ergebnisses auf den Hintergrundspeicher und somit \(B(R)\) Blockzugriffe.
		Das behandeln wir in \PipeliningKosten genauer.

		In der Praxis passen die ersten Blöcke aller Teillisten meistens in den Hauptspeicher.
		Sollte das nicht der Fall sein, kann eine weitere Merge-Phase angehängt werden.
	\end{solution}

\item Wie viele Blockzugriffe werden für die Durchführung eines Sort-Merge Joins von zwei Relationen \(R\) und \(S\) insgesamt benötigt, wenn zur Sortierung der Algorithmus aus Teilaufgabe \ref{item:sortierung}) und für die anschließende Durchführung des Merge Joins der in der Vorlesung (Folie~\MergeJoin) vorgestellte Algorithmus verwendet werden?

	\begin{solution}
		Die sortierten Relationen müssen schritthaltend jeweils einmal gelesen und das Ergebnis muss wieder abgelegt werden.
		Somit ergibt sich eine Gesamtzahl von \(6 (B(R) + B(S))\) Blockzugriffen.
		Äquivalent zur Sortierung spart die direkte Weiterverarbeitung der Join-Ergebnisse \(B(R) + B(S)\) Blockzugriffe.

		Für den Merge braucht man normal nur zwei Pufferkacheln.
		Sollte sich aber ein Wert eines Join-Attributs über mehrere Blöcke erstrecken, so müssen all diese Blöcke gleichzeitig in den Puffer gelesen werden.
		Für den Fall, dass dafür nicht genügend Kacheln zur Verfügung stehen, muss für die Tupel mit diesem Wert ein One-Pass Join oder ein Nested-Loop Join durchgeführt werden.
	\end{solution}

\item \label{item:kombiniert}Überlegen Sie, wie die Algorithmen aus den vorherigen beiden Teilaufgaben kombiniert werden können, um Blockzugriffe einzusparen.
	Welche Voraussetzungen müssen die am Join beteiligten Relationen dafür erfüllen?

	\begin{solution}
		\begin{enumerate}[1.]
		\item Führ die Schritte 1 bis 4 des Sortieralgorithmus für beide Relationen aus.
		\item	Lies den ersten Block jeder Teilliste beider Relationen in den Puffer.
		\item Such den niedrigsten Wert des Join-Attributs in allen Tupeln aller Blöcke im Puffer und join die Tupel mit dem gleichen Wert.
		\end{enumerate}

		Damit werden nur noch \(4 (B(R) + B(S))\) Blockzugriffe bzw.\ bei direkter Weiterverarbeitung \(3 (B(R) + B(S))\) Blockzugriffe benötigt.

		Voraussetzung für diesen Algorithmus ist, dass beide Relationen zusammen weniger Teillisten haben, als Hauptspeicherkacheln zur Verfügung stehen.
		Da jede Teilliste maximal \(M\) Blöcke enthält, muss gelten \( \left \lceil \frac{B(R)}{M} \right \rceil + \left \lceil \frac{B(S)}{M} \right \rceil < M\).
		Bei vielen gleichen Werten des Join-Attributs ist wie bei der nicht-optimierten Variante ein besonderes Vorgehen notwendig.
	\end{solution}
\end{enumerate}
