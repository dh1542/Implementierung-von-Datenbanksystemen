\section{Kosten der Operatorausführung}
\label{sec:kosten}

\begin{enumerate}[a)]

\item Bei einer kostenbasierten Optimierung stellen die Kosten die Größe dar, die minimiert werden soll.
	Welche Werte könnte man bei der Optimierung von Datenbankanfragen in die Kosten einfließen lassen?

	\begin{solution}
		Z.\,B.\ Anzahl Blockzugriffe auf den Hintergrundspeicher, CPU-Zeit, Hauptspeicherbedarf, \ldots
	\end{solution}

\item \label{item:kostenformeln} Wir nehmen ein einfaches Kostenmodell an, das nur die Lese- und Schreibvorgänge auf dem Hintergrundspeicher berücksichtigt.
	Stellen Sie dafür Kostenformeln für die Planoperatoren Selektion, Projektion, Nested-Loop Join, Sort-Merge Join und Hash Join auf.
	Gehen Sie davon aus, dass die Eingaben vom Hintergrundspeicher gelesen werden müssen und die Ergebnisse direkt ausgegeben werden.
	\(B(S)\) gibt die Anzahl der Blöcke an, aus denen die Relation \(S\) besteht.

	\begin{solution}
		\begin{tabular}{l l p{5cm}}
			\hline
			\textbf{Implementierung} & \textbf{Kosten}          &                                                         \\ \hline
			Selektion(S)             & $B(S)$                   &                                                         \\ \hline
			Projektion(S)            & $B(S)$                   &                                                         \\ \hline
			NestedLoopJoin(S, T)     & $B(S) + B(S) \cdot B(T)$ &                                                         \\ \hline
			SortMergeJoin(S, T)      & $3 (B(S) + B(T))$        & Unter den in \SortMergeJoin genannten Voraussetzungen   \\ \hline
			HashJoin(S, T)           & $B(S) + B(T)$            & Classic Hashing, wenn eine Relation in den Puffer passt \\ \hline
		\end{tabular}

		Die Kosten werden einheitslos angegeben, da sie nur zum Vergleich zwischen den Operatoren dienen.

		In der Literatur finden sich je nach verwendetem Kostenmodell und berücksichtigten Kostentypen unterschiedliche Formeln. Eine gute Herleitung von Kostenformeln steht bei Garcia-Molina et al.
	\end{solution}

\item \label{item:pipelining} Pipelining bezeichnet das direkte Hintereinanderausführen mehrerer Operatoren, ohne die Zwischenergebnisse erst komplett zu erzeugen und auf den Hintergrundspeicher zu schreiben.
	Welche der aus der Vorlesung und Übung bekannten Planoperatoren eignen sich für Pipelining gut, welche nicht?

	\begin{solution}
		Als Quelle für Pipelining können alle Operatoren dienen.
		Zu beachten ist dabei aber, dass vom Quelloperator belegte Pufferkacheln nicht für den Zieloperator zur Verfügung stehen.
		Werden Table Scan und Index Scan als eigenständige Operatoren behandelt, wie wir das im Folgenden tun, dann werden sie immer per Pipelining mit dem folgenden Operator verbunden.
		Ein Table-Scan-Operator ohne Pipelining wäre sinnlos, da er lediglich die Relation umkopieren würde.

		Als Ziel für Pipelining sind alle Operatoren gut geeignet, die die Quelle nur einmal durchlaufen müssen, beispielsweise Selektion, Projektion oder auch die äußere Relation des Nested-Loop Joins.

		Der Nested-Loop Join ist gleichzeitig auch ein Beispiel für einen Operator, der sich nicht gut als Ziel für Pipelining eignet.
		Wird die innere Relation nicht zwischengespeichert, muss sie für jeden Durchlauf neu erzeugt werden.
	\end{solution}

\item Wie ändern sich die Kostenformeln aus Teilaufgabe \ref{item:kostenformeln}), wenn Pipelining eingesetzt wird?
Verwenden Sie \(C(S)\) für die Kosten zur Erzeugung der Zwischenergebnisrelation \(S\).

	\begin{solution}
		Dient ein Operator A als Eingabe für einen weiteren Operator B, können dessen Kosten in der Kostenformel für B an der Stelle der Blockzugriffe zum Lesen der Eingaberelation angesetzt werden.
		Blockzugriffe für das Schreiben des Ergebnisses werden nicht benötigt, da das Ergebnis direkt an den nächsten Operator oder an den Anwender weitergegeben wird.
		Die Gesamtkosten einer Anfrage sind dann einfach die Kosten der Wurzel im Operatorbaum.

		Beispiel: Selektion(S) hat mit Pipelining Kosten von C(S). Das bedeutet, dass die Kosten eines Selektionsoperators gerade so hoch sind wie die Erzeugungskosten seiner Eingabedaten. Das liegt daran, dass wir nur E/A-Kosten berücksichtigen und die Selektion selbst keine Ein- oder Ausgabe vornimmt.

		Als Kostenformeln mit Pipelining ergeben sich:

		\begin{tabular}{l l}
			\hline
			\textbf{Implementierung} 	& \textbf{Kosten}\\
			\hline
			Selektion(S)						& $C(S)$\\
			\hline
			Projektion(S)						& $C(S)$\\
			\hline
			TableScan	(S)							& $B(S)$\\
			\hline
			IndexScan(S)							& $a \cdot \left \lceil{ B(S) \cdot \mathrm{Selektivit"atsfaktor} }\right \rceil $\\
			\hline
			NestedLoopJoin(S, T)			& $C(S) + B(S) \cdot C(T)$\\
			\hline
			SortMergeJoin(S, T)				& $C(S) + C(T) + 2 (B(S) + B(T))$\\
			\hline
			HashJoin(S, T) 						& $C(S) + C(T)$\\
			\hline
		\end{tabular}

		Beim Index Scan wurden hier die Kosten eines wahlfreien Blockzugriffs als a-mal so hoch wie die eines sequentiellen Blockzugriffs angesetzt.
	\end{solution}

\end{enumerate}
