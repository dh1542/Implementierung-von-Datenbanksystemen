\section{Serialisierbarkeit}
\label{serialisierbarkeit}

Geben Sie an, ob die folgenden Abläufe serialisierbar sind.
Skizzieren Sie dazu die zugehörigen Abhängigkeitsgraphen.
Notieren Sie dabei an jeder Kante den Namen des Datenobjekts, das für den Konflikt zwischen den beiden Transaktionen verantwortlich ist.

\begin{enumerate}[a)]

	\item \label{ser_schedule1} $r_2[B],~w_2[B],~c_2,~r_1[A],~r_3[A],~r_3[B],~w_1[A],~c_1,~w_3[A],~c_3$

\begin{solution}
	Nicht serialisierbar, da der Abhängigkeitsgraph zyklisch ist:

	\begin{tikzpicture}
			\node (T3) at (0,0) {$T_3$};
			\node (T1) [below of = T3, yshift = -0.5cm, xshift=-2cm] {$T_1$};
			\node (T2) [below of = T3, yshift = -0.5cm, xshift=+2cm] {$T_2$};
			\node [right of =T3, xshift=0.3cm, yshift=-0.5cm] {B};
			\node [below of = T3, xshift=-0.9cm] {A};
			\node [left of = T3, xshift=-0.25cm, yshift=-0.3cm] {A};

			\draw[-triangle 60] (T1) -- ($(T3.south) + (-0.2, 0.1)$);
			\draw[-triangle 60] (T2) -- ($(T3.south) + (0.3, 0.1)$);
			\draw[-triangle 60] ($(T3.west)$) -- ($(T1.north) + (-0,0)$);
	\end{tikzpicture}
\end{solution}

	\item $w_3[A],~w_3[B],~r_2[C],~r_1[A],~r_2[A],~c_2,~w_1[C],~c_1,~r_3[B],~c_3$

\begin{solution}
	Serialisierbar, da der Abhängigkeitsgraph azyklisch ist:

	\begin{tikzpicture}
		\node (T3) at (0,0) {$T_3$};
		\node (T1) [below of = T3, yshift = -0.5cm, xshift=-2cm] {$T_1$};
		\node (T2) [below of = T3, yshift = -0.5cm, xshift=+2cm] {$T_2$};
		\node [right of =T3, xshift=0.3cm, yshift=-0.5cm] {A};
		\node [left of = T3, xshift=-0.3cm, yshift=-0.5cm] {A};
		\node [below of = T3, yshift=-0.2cm] {C};

		\draw[-triangle 60] ($(T2.west)$) -- ($(T1.east)$);
		\draw[-triangle 60] ($(T3.south) + (0.3, 0.1)$) -- (T2);
		\draw[-triangle 60] ($(T3.south) + (-0.3, 0.1)$) -- (T1);
	\end{tikzpicture}

	Die einzige mögliche Reihenfolge ist also: $T_3 < T_2 < T_1$.
\end{solution}

\begin{note}
Es gibt drei Arten von Serialisierbarkeit.
Ohne Zusatz ist meist die Konfliktserialisierbarkeit gemeint (wie bei uns im Skript, siehe Definition von serialisierbaren Abläufen bzw.\ Äquivalenz von Abläufen).

Die Konfliktserialisierbarkeit basiert auf Konflikt-Äquivalenz, die verlangt, dass die Reihenfolge jeweils zweier in Konflikt stehender Operationen in zwei äquivalenten Abläufen gleich bleibt.
Besteht zwischen mehreren Abläufen Konflikt-Äquivalenz, dann müssen sie jeweils in ihren Abhängigkeitsgraphen zwischen den gleichen Transaktionen die gleichen gerichteten Kanten haben, weil im Abhängigkeitsgraphen per Definition die Konflikte zwischen Transaktionen gemäß ihrer Reihenfolge eingetragen werden.
Somit gilt: Zwei (konflikt-)äquivalente Abläufe haben identische Abhängigkeitsgraphen.

Die Gegenrichtung gilt offensichtlich nicht, da unterschiedliche Transaktionen den gleichen Graphen haben können.

Gilt es aber, wenn wir die gleichen Transaktionen nur unterschiedlich miteinander verzahnen? An dieser Stelle ist das egal.
\end{note}

\end{enumerate}
