\section{Optimierung}

Optimieren Sie die Ausführungspläne aus Aufgabe~\ref{plan}.
Nutzen Sie dazu die Regeln von Folien~\Operatorregel\ und die Heuristik auf den Folien~\Operatorheuristik.
Die aufgeführten Regeln für die algebraische Umformung sind aber nur Beispiele.
Formulieren Sie offensichtlich mögliche Umformungsmöglichkeiten selbst und nutzen Sie auch diese.

\begin{note}
Beides finden Sie als Referenz noch einmal am Ende dieses Übungsblatts.
\end{note}

\begin{solution}
\begin{enumerate}[a)]

\item Zur Heuristik:

\begin{itemize}
	\item Ausführungsplan beinhaltet nur Selektion mit einem Prädikat-Term $\rightarrow$ keine Separierung notwendig.
	\item Selektion und Kreuzprodukt können zusammengefasst werden (Umformungsregel 9).
\end{itemize}

Das ergibt folgenden optimierten Ausführungsplan:

\begin{center}
\begin{tikzpicture}
	\node (proj) at(0, 0) {\texttt{PROJ(\qquad, (e.fname, e.lname, s.fname, s.lname))} };
	\node (join) [below of =proj, xshift=-0.6cm] {\texttt{JOIN(\qquad, \qquad, (e.mgr = s.num))} };
	\node (rename_emp1) [below of = join, xshift=-2.5cm] {\texttt{RENAME(\qquad, e)}};
	\node (rename_emp2) [below of =join, xshift=1.5cm] {\texttt{RENAME(\qquad, s)}};
	\node (emp1) [below of = rename_emp1, xshift=0.3cm] {\texttt{employee}};
	\node (emp2) [below of = rename_emp2, xshift=0.3cm] {\texttt{employee}};

	\draw[-triangle 60] ($(rename_emp1.north west) + (0.8,0)$) -- ($(join.south) + (-1.8,0.1)$);
	\draw[-triangle 60] ($(rename_emp2.north west) + (0.8,0)$) -- ($(join.south) + (-0.8,0.1)$);
	\draw[-triangle 60] (emp1.north) -- +(0,0.5);
	\draw[-triangle 60] (emp2.north) -- +(0,0.5);
	\draw[-triangle 60] ($(join.north west) + (0.5,0)$) -- +(0,0.5);
\end{tikzpicture}
\end{center}

\item Die Selektion kann vor den Verbund geschoben werden (Umformungsregel 7).

\begin{center}
\begin{tikzpicture}
	\node (proj) at(0, 0) {\texttt{PROJ(\qquad, (e.fname, e.lname, e.addr))} };
	\node (join) [below of =proj, xshift=0.3cm] {\texttt{JOIN(\qquad, \qquad, (d.num = e.dep))} };
	\node (r_emp) [below of = join, xshift=-2.5cm] {\texttt{RENAME(\qquad, e)}};
	\node (emp) [below of = r_emp, xshift=0.3cm] {\texttt{employee}};
	\node (sel) [below of =join, xshift=3.5cm] {\texttt{SEL(\qquad, (d.name = 'Research'))} };
	\node (r_dept) [below of = sel, xshift=-1cm] {\texttt{RENAME(\qquad, d)} };
	\node (dept) [below of = r_dept, xshift=0.3cm] {\texttt{department} };

	\draw[-triangle 60] ($(r_dept.north west) + (0.8,0)$) -- +(0,0.5);
	\draw[-triangle 60] ($(r_emp.north west) + (0.8,0)$) -- ($(join.south) + (-1.9,0.1)$);
	\draw[-triangle 60] (dept.north) -- +(0,0.5);
	\draw[-triangle 60] ($(sel.north west) + (0.4,0)$) -- ($(join.south) + (-0.8,0.1)$);
	\draw[-triangle 60] (emp.north) -- +(0,0.5);
	\draw[-triangle 60] ($(join.north west) + (0.6,0)$) -- +(0,0.5);
\end{tikzpicture}
\end{center}

\item Hier kommen wir mit den Heuristiken und Umformungsregeln aus der Vorlesung nicht weiter.
	Da jedoch über den Primärschlüssel vom Department gruppiert wird, die Aggregation nur ein Count umfasst und die Vielfachheit nur aus dem Verbund mit dem Employee stammt, 
	kann man hier die Gruppierung am Join vorbei zum Employee ziehen.

\begin{tikzpicture}
	\node (proj) at (0, 0) {\texttt{PROJ(\qquad , (d.name))} };
	\node (join)[below of =proj, xshift=2.2cm] {\texttt{JOIN(\qquad, \qquad, (d.num = e.dep))} };
	\node (proj2)[below of =join, xshift=5cm] {\texttt{PROJ(\qquad , (d.name, d.num))} };
	\node (hav)[below of =join, xshift=-4cm] {\texttt{SEL(\qquad , (aggregierte $>$ 10))} };
	\node (group)[below of =hav, xshift=-0.3cm] {\texttt{GROUP(\qquad , (e.dep),}};
	\node (group2)[below of =group, xshift=3.5cm, yshift=0.5cm] {\texttt{((count(e.dep), aggregierte)))}};
	\node (renamed)[below of =proj2, xshift=-0.6cm] {\texttt{RENAME(\qquad, d)} };
	\node (renamee)[below of =group, yshift=-0.5cm, xshift =.5cm] {\texttt{RENAME(\qquad, e)} };
	\node (dept)[below of =renamed, xshift=0.3cm] {\texttt{department} };
	\node (emp)[below of =renamee, xshift=0.3cm] {\texttt{employee} };
	
	\draw[-triangle 60] (dept.north) -- +(0,0.5);
	\draw[-triangle 60] (emp.north) -- +(0,0.5);
	\draw[-triangle 60] ($(renamed.north west) + (0.8,0)$) -- +(0,0.5);
	\draw[-triangle 60] ($(renamee.north west) + (0.8,0)$) -- +(0,1);
	\draw[-triangle 60] ($(group.north west) + (0.65,0)$) -- +(0,0.5);
	\draw[-triangle 60] ($(hav.north west) + (0.7,0)$) -- ($(join.south west) + (1.5,0.1)$);
	\draw[-triangle 60] ($(proj2.north west) + (0.5,0)$) -- ($(join.south) + (-0.8,0.1)$);
	\draw[-triangle 60] ($(join.north west) + (0.5,0)$) -- +(0,0.5);

\end{tikzpicture}

\item

\begin{itemize}
	\item Die Selektion "`e.lname = 'Smith'"' kann in beiden Teilbäumen ganz nach unten gezogen werden (Umformungsregeln 6, 7, 8).
	\item Die Selektion und das Kreuzprodukt in den beiden Teilbäumen können jeweils zusammengefasst werden (Umformungsregel 9).
\end{itemize}

\begin{tikzpicture}
	\node (proj_top) at(0, 0) {\small{\texttt{PROJ(\hspace{0.5cm}, (e.lname))} } };
	\node (union) [below of =proj_top, xshift=0.6cm] {\small{\texttt{UNION(\qquad, \qquad)} } };

	\node (proj_left) [below of =union, xshift =-2.2cm] {\small{\texttt{PROJ(\hspace{0.5cm}, (p.num, e.lname))} } };
	\node (proj_right) [below of =union, xshift =5.7cm] {\small{\texttt{PROJ(\hspace{0.5cm}, (p.num, e.lname))} } };

	\node (join_left1) [below of =proj_left, xshift =0.2cm]  {\small{\texttt{JOIN(\hspace{0.5cm},\hspace{0.5cm}, (d.mgr = e.num))} } };
	\node (join_right1) [below of =proj_right, xshift =0.2cm] {\small{\texttt{JOIN(\hspace{0.5cm},\hspace{0.5cm}, (e.num = w.empl))} } };

	\node (join_left2) [below of =join_left1, xshift =1.7cm] {\small{\texttt{JOIN(\qquad, \qquad, (d.num = p.dep))} } };
	\node (join_right2) [below of =join_right1, xshift =1.7cm, yshift=-0.5cm] {\small{\texttt{JOIN(\qquad, \qquad, (p.num = w.proj))} } };

	\node (rename_dept_left) [below of =join_left2, xshift =-1.5cm, yshift=-0.5cm] {\small\texttt{RENAME(\qquad, d)} };
	\node (dept_left) [below of =rename_dept_left] {\small\texttt{department}};
	\node (rename_project_left) [below of =join_left2, xshift =1cm] {\small\texttt{RENAME(\qquad, p)}};
	\node (project_left) [below of =rename_project_left] {\small\texttt{project} };

	\node (rename_works_right) [below of =join_right2, xshift =-1.5cm, yshift =-0.5cm] {\small\texttt{RENAME(\qquad, w)}};
	\node (works_right) [below of =rename_works_right] {\small\texttt{works\_on} };
	\node (rename_project_right) [below of =join_right2, xshift =1cm] {\small\texttt{RENAME(\qquad, p)}};
	\node (project_right) [below of =rename_project_right] {\small\texttt{project} };


	\node (sel_left) [below of =dept_left, xshift =0cm] {\small{\texttt{SEL(\qquad, (e.lname = 'Smith'))} } };
	\node (sel_right) [below of =works_right, xshift =0cm] {\small{\texttt{SEL(\qquad, (e.lname = 'Smith'))} } };

	\node (rename_emp_left) [below of =sel_left, xshift =-1.5cm] {\texttt{RENAME(\qquad, e)}};
	\node (emp_left) [below of =rename_emp_left] {\small\texttt{employee} };
	\node (rename_emp_right) [below of =sel_right, xshift =-1cm] {\small\texttt{RENAME(\qquad, e)}};
	\node (emp_right) [below of =rename_emp_right] {\small\texttt{employee} };

	\draw[-triangle 60] (union) -- ($(proj_top.south) + (-0.8,0.1)$);
	\draw[-triangle 60] (proj_left) -- ($(union.south) + (-0.2,0.1)$);
	\draw[-triangle 60] (proj_right) -- ($(union.south) + (0.7,0.1)$);

	\draw[-triangle 60] (join_left1) -- ($(proj_left.south) + (-1.2,0.1)$);
	\draw[-triangle 60] (join_right1) -- ($(proj_right.south) + (-1.2,0.1)$);

	\draw[-triangle 60] (join_left2) -- ($(join_left1.south) + (-0.8,0.1)$);
	\draw[-triangle 60] (join_right2) -- ($(join_right1.south) + (-1,0.1)$);

	\draw[-triangle 60] (rename_dept_left) -- ($(join_left2.south) + (-1.7,0.1)$);
	\draw[-triangle 60] (rename_project_left) -- ($(join_left2.south) + (-0.8,0.1)$);

	\draw[-triangle 60] (rename_works_right) -- ($(join_right2.south) + (-1.9,0.1)$);
	\draw[-triangle 60] (rename_project_right) -- ($(join_right2.south) + (-0.8,0.1)$);

	\draw[-triangle 60] ($(sel_left.north) + (-2.4, 0 )$) -- ($(join_left1.south) + (-1.55,0.1)$);
	\draw[-triangle 60] ($(sel_right.north) + (-2, 0 )$) -- ($(join_right1.south) + (-1.75,0.1)$);

	\draw[-triangle 60] (rename_emp_left) -- ($(sel_left.south) + (-1.7,0.1)$);
	\draw[-triangle 60] (rename_emp_right) -- ($(sel_right.south) + (-1.7,0.1)$);
	\draw[-triangle 60] (emp_left) -- ($(rename_emp_left.south) + (0.3,0.1)$);
	\draw[-triangle 60] (dept_left) -- ($(rename_dept_left.south) + (0.3,0.1)$);
	\draw[-triangle 60] (project_left) -- ($(rename_project_left.south) + (0.3,0.1)$);
	\draw[-triangle 60] (emp_right) -- ($(rename_emp_right.south) + (0.3,0.1)$);
	\draw[-triangle 60] (works_right) -- ($(rename_works_right.south) + (0.3,0.1)$);
	\draw[-triangle 60] (project_right) -- ($(rename_project_right.south) + (0.3,0.1)$);
	\end{tikzpicture}

\paragraph{\color{solutioncolor}Ergänzung:}
In der Literatur ist es üblich, Operatoren mit griechischen Buchstaben darzustellen.
Diese Darstellung sollte vor allem den Lehramtsstudierenden bekannt sein.
Der eben erstellte Baum würde dann folgendermaßen aussehen:

\begin{center}
\begin{tikzpicture}
    \node (proj_top) at(0, 0) {$\pi_{\text{e.lname}}$ };
    \node (union) [below of =proj_top] {$\cup$} ;

    \node (proj_left) [below of =union, xshift =-4.0cm] {$\pi_{\text{p.num, e.lname}}$ };
    \node (proj_right) [below of =union, xshift =4.0cm] {$\pi_{\text{p.num, e.lname}}$ };

    \node (join2_left) [below of =proj_left] {$\Bowtie_{\text{d.mgr = e.num}}$};
    \node (join2_right) [below of =proj_right] {$\Bowtie_{\text{e.num = w.empl}}$};

    \node (sel_left) [below of =join2_left, xshift = -1.9cm] {$\sigma_{\text{e.lname = 'Smith'}}$};
    \node (sel_right) [below of =join2_right, xshift = -1.9cm] {$\sigma_{\text{e.lname = 'Smith'}}$};

    \node (join1_left) [below of =join2_left, xshift = 1.8cm] {$\Bowtie_{\text{d.num = p.dep}}$};
    \node (r_dep_left) [below of =join1_left,xshift = -1.1cm] {$\rho_{\text{d}}$};
    \node (r_project_left) [below of =join1_left,xshift = 1.1cm] {$\rho_{\text{p}}$};
    \node (r_emp_left) [below of =sel_left] {$\rho_{\text{e}}$};
    \node (dep_left) [below of =r_dep_left] {\small{\texttt{department}}};
    \node (project_left) [below of =r_project_left] {\small{\texttt{project}}};
    \node (emp_left) [below of =r_emp_left] {\small{\texttt{employee}}};

    \node (join1_right) [below of =join2_right, xshift = 1.8cm] {$\Bowtie_{\text{p.num = w.proj}}$};
    \node (r_works_right) [below of =join1_right,xshift = -1.1cm] {$\rho_{\text{w}}$};
    \node (r_project_right) [below of =join1_right,xshift = 1.1cm] {$\rho_{\text{p}}$};
    \node (r_emp_right) [below of =sel_right] {$\rho_{\text{e}}$};
    \node (works_right) [below of =r_works_right] {\small{\texttt{works\_on}}};
    \node (project_right) [below of =r_project_right] {\small{\texttt{project}}};
    \node (emp_right) [below of =r_emp_right] {\small{\texttt{employee}}};

    \draw (r_emp_left) -- ($(sel_left.south)$);
    \draw (emp_left) -- ($(r_emp_left.south)$);
    \draw (sel_left) -- ($(join2_left.south)$);
    \draw (join2_left) -- ($(proj_left.south)$);
    \draw (proj_left) -- ($(union.south)$);

    \draw (r_dep_left) -- ($(join1_left.south)$);
    \draw (r_project_left) -- ($(join1_left.south)$);
    \draw (dep_left) -- ($(r_dep_left.south)$);
    \draw (project_left) -- ($(r_project_left.south)$);
    \draw (join1_left) -- ($(join2_left.south)$);

    \draw (r_emp_right) -- ($(sel_right.south)$);
    \draw (emp_right) -- ($(r_emp_right.south)$);
    \draw (sel_right) -- ($(join2_right.south)$);
    \draw (join2_right) -- ($(proj_right.south)$);
    \draw (proj_right) -- ($(union.south)$);

    \draw (r_works_right) -- ($(join1_right.south)$);
    \draw (r_project_right) -- ($(join1_right.south)$);
    \draw (works_right) -- ($(r_works_right.south)$);
    \draw (project_right) -- ($(r_project_right.south)$);
    \draw (join1_right) -- ($(join2_right.south)$);

    \draw (union) -- ($(proj_top.south)$);
\end{tikzpicture}
\end{center}

\end{enumerate}
\end{solution}

\begin{note}
    Anmerkung: Diese Darstellung könnte vor allem für Lehramtsstudierende wichtig sein,
    da sie prinzipiell im Staatsexamen erwartet werden kann.
\end{note}
