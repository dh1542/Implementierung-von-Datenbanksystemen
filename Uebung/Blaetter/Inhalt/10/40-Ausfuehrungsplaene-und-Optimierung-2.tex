\section{Ausführungspläne und Optimierung 2}

Gegeben sind folgende aus Aufgabe \ref{plan} bekannte Relationen:

\texttt{Employee (\underline{num}, fname, lname, addr, mgr[employee], dep[department])}

\texttt{Department (\underline{num}, name, mgr[employee])}

\texttt{Project (\underline{num}, dep[department], mgr[employee])}

\texttt{Works\_on (\underline{proj[project], empl[employee]})}

Setzen Sie die folgenden SQL-Anweisungen in nicht-optimierte Operatorgraphen um.
Verwenden Sie dafür die Baum-Notation von Vorlesungsfolie~\Operatorgraph.
Optimieren Sie die Anweisungen anschließend und zeichnen Sie jeweils den optimierten Operatorgraph.

\cprotEnv
\begin{normalText}
\begin{enumerate}[a)]
	\item
\begin{lstlisting}
-- Alle MitarbeiterInnen mit Vorgesetzten in anderen
-- Departments
SELECT e.fname, e.lname
FROM   employee e, employee s
WHERE  e.mgr = s.num
AND    NOT e.dep = s.dep
\end{lstlisting}

\cprotEnv
\begin{note}
Nicht optimiert:
\begin{lstlisting}
- PROJ( , (e.fname, e.lname))
  - SEL( , (e.mgr = s.num AND NOT e.dep = s.dep))
    - CROSS(employee e, employee s)
\end{lstlisting}
Optimiert:
\begin{lstlisting}
- PROJ( , (e.fname, e.lname))
  - JOIN( , ,  (e.mgr = s.num AND NOT e.dep = s.dep))
    - employee e
    - employee s
\end{lstlisting}
Regel 9
\end{note}

\item

\begin{lstlisting}
-- Alle Vorgesetzen, die ein Department leiten, zu dem sie
-- nicht selbst gehoeren
SELECT   s.fname, s.lname
FROM     employee s
JOIN     department d
ON       d.mgr = s.num
JOIN     employee e
ON       e.mgr = s.num
WHERE    NOT d.num = s.dep
GROUP BY s.fname, s.lname, s.num
\end{lstlisting}
\cprotEnv
\begin{note}
Nicht optimiert:
\begin{lstlisting}
-PROJ( , (s.fname, s.lname))
  -GROUP( , (s.fname, s.lname, s.num))
    -SEL( , (NOT d.num = s.dep))
      -JOIN(employee e, , (e.mgr = s.num))
        -JOIN(employee s, department d, (d.mgr = s.num))
\end{lstlisting}
  Optimiert: Sel durch äußeren Join ziehen, dann Sel und Join verschmelzen.
  e vor dem Join auf mgr projizieren.\\
\begin{lstlisting}
-PROJ( , (s.fname, s.lname))
  -GROUP( , (s.fname, s.lname, s.num))
    -JOIN( , (e.mgr = s.num))
      -PROJ(employee e, (e.mgr))
      -JOIN( , , (d.mgr = s.num AND NOT d.num = s.dep))
        -employee s
        -department d

\end{lstlisting}
Regeln 1, 7, 9, 4, 9, 6
\end{note}

\item

	\begin{lstlisting}
-- Pro Department die durchschnittliche Anzahl der
-- Mitarbeiter der von diesem Department durchgefuehrten
-- Projekte fuer Departments, die von einem Mitarbeiter
-- mit Nachnamen Smith geleitet werden
SELECT   AVG(a.anzahl) as durchschnitt, d.name
FROM     (
          SELECT   count(*) as anzahl, proj
          FROM     works_on
          GROUP BY proj
         ) a
JOIN     project p
ON       a.proj = p.num
JOIN     department d
ON       d.num = p.dep
JOIN     employee m
ON       m.num = d.mgr
GROUP BY m.lname, d.num, d.name
HAVING   m.lname = 'Smith'
	\end{lstlisting}

\begin{note}
	Nicht optimiert:
	\begin{equation*}
\begin{split}
	 & \pi_{durchschnitt, d.name}( \quad )                                                     \\
	 & \quad \sigma_{m.lname = 'Smith'}(\quad)                                                 \\
	 & \qquad \gamma_{(m.lname, d.num, d.name), avg(a.anzahl) \rightarrow durchschnitt}(\quad) \\
	 & \qquad\quad \underset{a.proj = p.num}{\bowtie}                                          \\
	 & \qquad \quad  \underset{d.num = p.dep}{\bowtie}                                         \\
	 & \qquad \quad \underset{m.num = d.mgr}{\bowtie}                                          \\
	 & \qquad\qquad \rho_{a}(\pi_{anzahl, proj}(\quad )                                        \\
	 & \qquad\qquad\quad\gamma_{(proj), count(*) \rightarrow anzahl}(works\_on))               \\
	 & \qquad\qquad \rho_{p}(project)                                                          \\
	 & \qquad\qquad \rho_{d}(department)                                                       \\
	 & \qquad\qquad \rho_{m}(employee)
\end{split}
\end{equation*}

Optimierungen: Sicherlich vieles möglich, offensichtlich: HAVING durch GROUP ziehen, GROUP um lname kürzen.
\end{note}
\end{enumerate}
\end{normalText}
