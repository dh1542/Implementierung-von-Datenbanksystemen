\section{Programmieraufgabe 7: Planoperatoren}

\subsection{Aufgabenstellung}
\begin{enumerate}
	\item Implementieren Sie 6 Klassen, die die Schnittstelle \beamertxt{\linebreak}\texttt{idb.module.Module} implementieren.
		Beachten Sie die Dokumentation der Methoden in der Schnittstelle.
		Diese 6 Klassen entsprechen den Planoperatoren \textit{Projektion, Rename, Selektion, Kreuzprodukt, OrderBy, Table Scan}.
		Nutzen Sie für alle Planoperatoren, die Daten aus einem anderen Operator erhalten, ein \texttt{idb.module.Module} als Quelle dieser Daten, wie auch in den entsprechenden Methoden in \texttt{idb.construct.Util} vorgesehen.
		Bei dem Kreuzprodukt entsprechen die beiden Strings Präfixe für den Namen der jeweiligen Seite. \texttt{leftS} kann auch auf \texttt{null} gesetzt werden um Vielfachkreuzprodukte umsetzen zu können. In diesem Fall werden Attributnamen aus dieser Seite ohne Präfix übernommen.
		Für OrderBy ist ein Komparator bereits in \texttt{idb.module.Util} gegeben. Die Syntax \texttt{String... arg} entspricht auf der Callee-Seite einem Array und ist damit mit dem Parameter in \texttt{idb.module.Util} ausgelegt.
		Achten Sie bei OrderBy darauf, das nächste Element im \texttt{pull()} mit Bubblesort in \textit{O(n)} zu finden.
		Implementieren Sie keinen anderen (Pufferlastigeren) Sortieralgorithmus.
		Tragen Sie die entsprechenden Konstruktoren in \texttt{idb.construct.Util} ein.
	\item Implementieren Sie zusätzlich unter \textit{Store} einen Planoperator, der eine \linebreak \texttt{idb.record.SeqRecordFile} nutzt, um alle Daten des darin eingebauten Operators beim ersten \texttt{pull()} in eine neu angelegte Datei zu speichern und ab dann diese auszulesen. Achten Sie darauf, diese Datei mit \texttt{java.io.File.createTempFile} anzulegen und im \texttt{close} wieder zu löschen.
		Tragen Sie den entsprechenden Konstruktor in \texttt{idb.construct.Util} ein.
	\item Implementieren Sie außerdem unter \textit{Generate} einen Planoperator, mit dessen Hilfe weitere Attribut-Wert Kombinationen (als \texttt{NamedCombinedRecord}) zu den bereits enthaltenen Informationen erzeugt werden können (z.B. bei der Berechnung und Filterung von Strings nach ihrer Länge). Die von der übergebenen Funktion zurückgegebene \texttt{NamedCombinedRecord} sind als Zusätze und nicht als Ersatz der Tupel zu sehen (Hinweis: \texttt{NamedCombinedRecord::combine}).
		Tragen Sie den entsprechenden Konstruktor in \texttt{idb.construct.Util} ein.
	\item Implementieren Sie die Funktion \texttt{idb.construct.Util.delete}, mit deren Hilfe Daten aus einer TID-File gelöscht werden sollen.
		Achten Sie darauf, alle Module ordnungsgemäß zu schließen.
		Bei einer eventuell auftretenden Exception muss Ihre Datenbank sich nicht korrekt verhalten.
	\item Implementieren Sie abschließend \textit{GroupBy} mit Verwendung der bereits existierenden \textit{OrderBy, Projektion}. \texttt{String ...vars} geht dabei direkt in den \textit{OrderBy} über (mittels \texttt{generateOrder(/*TODO*/, /*TODO*/, vars)}).
		\texttt{funcs} entspricht dabei den Aggregationsfunktionen für die Gruppierung.
		Dabei entspricht jeder Eintrag in dieser Liste einer Aggregierung (z.B. \textit{max(MatrNr)}).
		Das Tripel besteht aus den folgenden drei Teilen:
		\begin{enumerate}
			\item Initialer Wert als \texttt{NamedCombinedRecord} für eine leere Gruppierung.
			\item Name des neu erzeugten Attributes als String.
			\item Funktion um aus zwei (Zwischen-)werten einen gemeinsamen Wert zu bilden. (z.B. bei Count: sowas wie \texttt{return a + b}, jedoch für NamedCombinedRecord).
		\end{enumerate}
		Zur Erzeugung dieser Tripels gibt es \texttt{count}, \texttt{min}, \texttt{max} in \texttt{idb.construct.Util}.
		\textit{Seitenbemerkung, für Interessierte: Diese Darstellung von Funktionen funktioniert nur für eine reduzierte Menge an Funktionen wie count, max, min, sum.
		Für avg ist sie ungeeignet, avg muss über die Eigenschaft sum/count umgesetzt werden.}
		Tragen Sie den entsprechenden Konstruktor in \texttt{idb.construct.Util} ein.
	\item Sorgen Sie dafür, dass Sie alle Tests aus den Klassen in \texttt{tests/ProjTests} erfüllen.
	Sie können diese Testfälle mit \lstinline|ant Meilenstein7| ausführen.
	Die einzelnen Unterpunkte hier lassen sich mit \lstinline|ant Meilenstein7a| bis \lstinline |ant Meilenstein7e| ausführen (mit Ausnahme von \lstinline|ant Meilenstein7b|, der nur durch \lstinline|ant Meilenstein 7d| abgedeckt wird).
	Sollten diese Meilensteine 7a bis 7e fehlen, aktualisieren Sie bitte Ihre Version auf die aktualisierten Build-Files der Angabe.
	\item Achten Sie bei allen Modulen darauf, dass die Konstruktion günstig ist und nicht schon \texttt{pull, read} oder ähnliche Funktionen aufruft.
		Sollten Sie Daten haben, die durch einen Aufruf dieser Funktionen initialisiert werden würden, markieren Sie sich den Zustand als unbearbeitet und führen Sie entsprechende Initialisierungen erst beim ersten \texttt{pull} aus.
	\item Die Abgabe auf GitLab erfolgt zeitgleich mit der Abgabe der Zusatzaufgaben des nächsten Übungsblattes auf StudOn. Markieren Sie hierfür Ihre Abgabe mit dem Tag "`Aufgabe-7"'.
\end{enumerate}
