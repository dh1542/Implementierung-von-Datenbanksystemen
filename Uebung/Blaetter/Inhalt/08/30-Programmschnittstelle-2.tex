\section{Programmschnittstelle 2}

Gegeben sei folgende SQL-Abfrage:

\texttt{SELECT Kunden.nachname, Kunden.vorname, auftraege.status, auftraege.summe \\
	FROM Kunden JOIN auftraege ON Kunden.id = auftraege.kunde \\
	WHERE auftraege.id = 42;}

An der internen Satzschnittstelle stehen folgende Objekte zur Verfügung:
\begin{itemize}
	\item eine Datei \emph{Kunden}, die die Kundendatensätze in Form von Tupeln der Relation enthält;
	\item eine Datei \emph{Auftraege}, die die Auftragsdatensätze in Form von Tupeln der Relation enthält;
	\item eine Datei \emph{Kundennummer-Index}, die eine Kundennummer auf eine Satzadresse abbildet;
	\item eine Datei \emph{Auftragsnummer-Index}, die eine Auftragsnummer auf eine Satzadresse abbildet.
\end{itemize}
Jede Kundennummer und jede Auftragsnummer ist eindeutig.

\begin{enumerate}[a)]
	\item Schreiben Sie ein Programm, das die o.\,g.\ SQL-Anfrage mit den Aufrufen der internen Satzschnittstelle realisiert. Verwenden Sie eine Sprache Ihrer Wahl oder auch Pseudocode.

	\begin{note}
	\texttt{
	\begin{tabbing}
		\hspace*{1cm}\=\hspace*{1cm}\= \kill
		KeyedRecordFile kundenindex = new KeyedRecordFile("Kundennummer-Index", "r"); \\
		KeyedRecordFile auftraegeindex = new KeyedRecordFile("{}Auftragsnummer-Index", "r"); \\
		if(!auftraegeindex.contains(42))\\
		\> abort;
		TID auftraegetid = auftraegeindex.readKey(42); \\
		DirectRecordFile auftraegesaetze = new DirectRecordFile("{}Auftraege", "r"); \\
		Record auftraegeergebnis = auftraegesaetze.read(auftraegetid); \\
		Int kundennr = auftraegeergebnis.get(kunde); \\
		TID kundentid = kundenindex.readKey(kundennr); \\
		DirectRecordFile kundensaetze = new DirectRecordFile("Kunden", "r"); \\
		Record kundenergebnis = kundensaetze.read(kundentid); \\
		print(kundenergebnis.read(nachname).toString());\\
		print(kundenergebnis.read(vorname).toString());\\
		print(auftraegeergebnis.read(status).toString());\\
		print(auftraegeergebnis.read(summe).toString());\\
	\end{tabbing}
	}
	\end{note}

	\item Wie würde das Programm aussehen, wenn kein Index zur Verfügung stünde?

	\begin{note}
		\texttt{
			\begin{tabbing}
			\hspace*{1cm}\=\hspace*{1cm}\= \kill
			DirectRecordFile kundensaetze = new DirectRecordFile("Kunden", "r"); \\
			DirectRecordFile auftraegesaetze = new DirectRecordFile("{}Auftraege", "r"); \\
			Record kundenergebnis; \\
			Record auftraegeergebnis; \\
			Bool gefunden = false;\\
			while (auftraegesaetze.hasNext()) \{  \\
			\> auftraegeergebnis = auftraegesaetze.next(); \\
			\> if (auftraegeergebnis.get(id) == 42) \{ \\
			\> \> gefunden = true;\\
			\> \> break; //Abbruch, jede Auftragsnummer existiert 1x \\
			\> \} \\
			\} \\
			if not gefunden \{ \\
			\> abort; //Abbruch, Auftrag nicht gefunden \\
			\} \\
			gefunden = false; \\
			while (kundensaetze.hasNext()) \{   \\
			\> kundenergebnis = kundensaetze.next(); \\
			\> if (kundenergebnis.get(id) == auftraegeergebnis.get(kunde)) \{ \\
			\> \> gefunden = true;\\
			\> \> break; //Abbruch, jede Kundennummer existiert 1x \\
			\> \} \\
			\} \\
			if not gefunden \{ \\
			\> abort; //Abbruch, Kunde nicht gefunden \\
			\} \\
			print(kundenergebnis.get(nachname).toString()); \\
			print(kundenergebnis.get(vorname).toString()); \\
			print(auftraegeergebnis.get(status).toString()); \\
			print(auftraegeergebnis.get(summe).toString()); \\
			\end{tabbing}
		}
	\end{note}

\end{enumerate}

