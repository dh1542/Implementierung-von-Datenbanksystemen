\section{Programmieraufgabe 5: ClockImpl}

\subsection{Aufgabenstellung}
\begin{enumerate}
	\item Implementieren Sie eine Klasse, die die Schnittstelle \beamertxt{\linebreak}\texttt{idb.buffer.DBBuffer} implementiert.
		Beachten Sie die Dokumentation der Methoden in der Schnittstelle. Beachten Sie, dass diese Aufgabe den Übungsaufgaben zum Thema Pufferverwaltung mit Clock entspricht.
		Ihre Implementierung soll dieses Verhalten umsetzen.
	\item Tragen Sie den Konstruktor Ihrer Klasse in \texttt{idb.construct.Util} in der Methode \texttt{generateClockBuffer()} ein.
		Die Funktion nimmt als Parameter die Pagesize und die Anzahl an Kacheln in dem ClockBuffer.
	\item Sorgen Sie dafür, dass Sie alle Tests aus der Klasse \texttt{ClockBufferTests} erfüllen.
	Sie können diese Testfälle mit \lstinline|ant Meilenstein5| ausführen.
	\item Die Abgabe auf GitLab erfolgt zeitgleich mit der Abgabe der Zusatzaufgaben des nächsten Übungsblattes auf StudOn. Markieren Sie hierfür ihre Abgabe mit dem Tag "`Aufgabe-5"'.
\end{enumerate}

\subsection{Hinweise}
\begin{itemize}
	\item Die Klasse \texttt{idb.buffer.Page} zur Verwaltung von Seiten ist schon gegeben. Nutzen Sie diese bei der Implementierung.
	\item Das \texttt{override} Flag in \texttt{idb.buffer.Page::clear} wird verwendet, um den ByteBuffer nach dem Zurückschreiben auf die Platte zu überschreiben.
		Dies ist im Falle des ClockBuffers nicht nötig.
	\item Achten Sie darauf, eine Seite erst zu leeren (\texttt{clear}), sobald Sie nach dem Verfahren Clock entschieden haben, die Seite zu verdrängen und nicht schon beim \texttt{unfix()}.
	\item Eine Klasse zur Verwaltung von Blocknummer und \texttt{BlockFile} ist in \texttt{idb.buffer.PageDescriptor} schon gegeben. Diese Klasse kann mittels \texttt{equals} auf Gleichheit geprüft werden und kann als Schlüssel in einer Hash-Struktur aus der Java Standardbibliothek verwendet werden.
	\item Achten Sie darauf, in der Methode \texttt{flush} nur Seiten zu leeren, die nicht gerade gefixt sind. Achten Sie außerdem darauf, dass die Seiten nach dem Aufruf der Methode \texttt{flush} immer noch im Puffer verbleiben und nur ihre Änderungen in die Blockfiles geschrieben werden.
\end{itemize}
