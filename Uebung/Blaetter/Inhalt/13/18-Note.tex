\begin{note}
Sollte das Übungsblatt zu lang sein, kann man je nach Verlauf der Übung
\begin{enumerate}
  \item eine der Wiederholungsfragen weglassen (sind sowieso 1:1 aus den Vorlesungsfolien übernommen)
  \item die kostenbasierte Optimierung abkürzen: Die Studenten erkennen bereits in Teilaufgabe 2a), dass ein Hash-Join schneller als ein Sort-Merge-Join und dieser wiederum schneller als ein Nested-Loop-Join ist. Deshalb kann man sich in Teilaufgabe b) immer darauf beschränken, nur die schnellste mögliche Join-Implementierung zu berechnen. Dadurch fallen auch fast alle Abschätzungen von Zwischenergebnisgrößen weg, was den zeitlichen Umfang deutlich begrenzt.
\end{enumerate}
\end{note}
