\section{Dateizugriff}
Schreiben Sie im Pseudocode die Operation, die einen Block aus einer Datei liest.
Der Dateikatalog soll als Extent-Tabelle gemäß Folie~\KatalogeintraegeVier~im Anhang realisiert sein.
Die Datei kann als bereits geöffnet angenommen werden.
Die Kommunikation mit der Festplatte sei über die Funktion \texttt{readBlock(int CylinderNo, int TrackNo, int SlotNo, ByteBuffer bb)} der Klasse \texttt{Device} zu gestalten.
Die Signatur der Methode \texttt{read} is Folgende: \texttt{void read(int BlockNo, ByteBuffer bb) throws IOException{}}.

\cprotEnv
\begin{note}
\begin{lstlisting}[]
/* Operation
 * int BlockFile::read ( int BlockNo, char *BlockBuffer )
 * wie auf Folie(*@~\textit{\SchreibUndLese}@*)
 */
void read(int BlockNo, ByteBuffer bb) throws IOException {
/* Lesen der Extent-Tabelle auf dem Katalog-Eintrag der
 * (bereits geöffneten) Datei:
 * Extent[i] (Slot-Adresse; AnzSlots)
 * AnzSlots = Zahl der von dieser Adresse an
 * sequenziell belegten Slots
 */
int LocalBlockNo = BlockNo; bool found=false;
// Finde passenden Extent
for(int i = 0; i < Extent.length; i++) {
  if(LocalBlockNo > Extent[i].AnzSlots ) {
    LocalBlockNo -= Extent[i].AnzSlots;
  } else {found=true; break;}
}
if(!found) throw new IOException();
// Startkoordinaten des Extents
int CylinderNo = Extent[i].Slot-Adresse.Zylinder;
int TrackNo = Extent[i].Slot-Adresse.Spur;
int SlotNo = Extent[i].Slot-Adresse.Slot;
// Bewege Koordinaten entsprechend weiter
SlotNo += LocalBlockNo;
/* Überlaufbehandlung: */
while(SlotNo > SlotsPerTrack) {
  SlotNo -= SlotsPerTrack; TrackNo++;
}
while(TrackNo > TracksPerCylinder) {
  TrackNo -= TracksPerCylinder; CylinderNo++;
}
if(CylinderNo > MaxCylinder) throw new IOException();
int Status = Device.readBlock(CylinderNo, TrackNo, SlotNo, bb )
if (Status != 0)  throw new IOException();
return
}
\end{lstlisting}
\end{note}
