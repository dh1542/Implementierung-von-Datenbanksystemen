\section{Fragen zur Vorlesung}
\begin{enumerate}[a)]
	\item Wozu dient die Schichtenbildung in der Softwarearchitektur?

	\begin{solution}
	Auf den Punkt gebracht: Zur Strukturierung komplexer Software-Systeme.

	Vorteile von Schichtenbildung (Folie~\SchichteVorteil):
	\begin{itemize}
		\item Höhere Ebenen werden einfacher, weil sie tiefere Ebenen benutzen können
		\item Änderungen auf höheren Ebenen haben keinen Einfluss auf tiefere Ebenen
		\item Höhere Ebenen können abgetrennt werden, tiefere Ebenen bleiben trotzdem funktionsfähig (und damit wieder verwendbar)
		\item Tiefere Ebenen können getestet werden, bevor die höheren Ebenen lauffähig sind
		\item Änderungen auf tieferen Ebenen bei Beibehalten der Schnittstellen benötigen keine Änderungen der höheren Schichten.
		(Bsp: Bei Einbau einer neuen, größeren Festplatte muss nicht das Programm umgeschrieben werden)
	\end{itemize}
	\end{solution}


	\item Wodurch kann eine Schicht beschrieben werden?

	\begin{solution}
	Funktional ("`was tut eine Schicht?"', Folie~\SchichteAufgabe):
	\begin{itemize}
		\item Realisiert einen bestimmten Dienst, den sie an der Schnittstelle "`nach oben"' darüber liegenden Schichten zur Verfügung stellt
		\item Nimmt Dienste darunter liegender Schichten in Anspruch
		\item Verbirgt die darunter liegenden Schichten vollständig und muss daher alle erforderlichen Funktionen anbieten
	\end{itemize}

	Strukturell ("`was sieht man an der Schnittstelle?"'):
	\begin{itemize}
		\item Durch die ausgetauschten Datenobjekte und die angebotenen Funktionen
	\end{itemize}
	\end{solution}


	\item Welche Beispiele für schichtenartig aufgebaute Software-Systeme außer Datenbanksystemen kennen Sie noch?

	\begin{solution}
	\begin{itemize}
		\item ISO/OSI-Referenzmodell für Kommunikationsprotokolle
		\item Linux-Kernel (\url{http://www.makelinux.net/kernel_map/})
		\item Firefox-Browser (\url{https://www.researchgate.net/figure/Mozilla-Firefox-architecture-The-User-Interface-is-split-over-two-subsystems-allowing_fig4_325076604})
		\item Behauptung: fast jede große Software ist in Schichten aufgebaut
	\end{itemize}
	\end{solution}


	\item Was versteht man unter der Datenunabhängigkeit einer Anwendung?

	\begin{solution}
	Datenunabhängigkeit einer Anwendung (Folie~\Datenunabhaengig): Speichern und Wiedergewinnen (= Auffinden und Aushändigen) von persistenten Daten ohne Kenntnis der Details der Speicherung.
	Kommt in der Informatik häufig vor, aber unter anderen Namen: Information Hiding, Datenabstraktion, Datenkapselung usw.
	\end{solution}

	\item Welche \deepen Vor- und Nachteile hat die blockorientierte Dateischnittstelle im Vergleich zur direkten Adressierung über Zylinder, Spur und Sektornummer?

	\begin{note}
		Folie~\LogischeSpeichergeraete, nur umgekehrt.
		Zusätzlich:~\Geraeteunabhaengigkeit
	\end{note}

	\item Welche \deepen Verwaltungsdatenstrukturen benötigt die blockorientierte Dateischnittstelle?

	\begin{note}
		Folie~\Dateikatalog
	\end{note}
\end{enumerate}
\beamertxt{\pagebreak}

