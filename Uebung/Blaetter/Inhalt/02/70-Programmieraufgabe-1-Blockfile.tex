\section{Programmieraufgabe 1: Blockfile}

\subsection{Generelles}
Sinn der Programmieraufgabe ist es, im Laufe des Semester eine single-threaded Datenbank selbst zu implementieren.
Die Vorlage, die Sie auf GitLab\footnote{\url{https://gitlab.cs.fau.de/idbprog/vorlage}} herunterladen können, beinhaltet einige Schnittstellen, Hilfsklassen und Testfälle.
Achten Sie darauf, Ihre Implementierung verständlich zu halten, da Sie ggf. im Laufe des Semesters mehrfach bereits gelöste Aufgaben nach versteckten Fehlern durchsuchen müssen.
Wir empfehlen eine Gruppengröße von 2-3 Personen.

Zur Abgabe verwenden wir Git-Repositorys des GitLab der Informatik\footnote{\url{https://gitlab.cs.fau.de}}, auf welche sowohl Sie, als auch die Tutoren Zugriff haben.
Dies erstellen wir für Sie, loggen Sie sich hierfür erstmalig auf der GitLab-Weboberfläche ein und schreiben Sie eine E-Mail mit den Namen und GitLab-Kennungen aller Gruppenmitglieder an \href{mailto:demian.voehringer@fau.de}{demian.voehringer@fau.de}.
Wir empfehlen, dieses auch für Ihre gemeinsame Arbeit zu verwenden.

Die Programmieraufgabe verwendet \textit{ant} und \textit{junit5}.
Sie sollte im CIP direkt funktionieren.
Der CIP dient auch als Referenzsystem, auf dem Ihre Abgabe laufen sollte. %muss.
Zur Übersetzung und Ausführung der \texttt{idb.Main} führen Sie \texttt{ant} in Toplevel aus, für die Testfälle \texttt{ant test}.

Zögern Sie nicht, uns bei Unklarheiten eine Frage im Forum auf StudOn zu stellen.
Mithilfe Ihrer Rückmeldung können die Aufgaben verbessert und Unklarheiten beseitigt werden.
Bitte halten Sie für solche Fragen Ihr Git-Repository auf dem aktuellen Stand, sodass für uns die Möglichkeit besteht, Ihren Code zu betrachten und Ihnen Rückmeldungen zu geben.
%Zusätzlich können Sie sich an die Intensivierungsübung zur Klärung von Fragen wenden.

Die genauen Definitionen sind in den Schnittstellen und durch die Testfälle gegeben.
Ziel ist es, bis zum Ende des Semesters alle Testfälle zu bestehen, wobei ggf. weitere Testfälle während des Semesters entwickelt und an Sie verteilt werden können.
%Zusätzlich schauen wir Stichprobenweise auf die Implementierung, um ein reines Implementieren bis alle Tests klappen zu vermeiden.
%Versuchen Sie auch deswegen einen verständlichen Stil einzuhalten.


Die Aufgaben bestehen aus der Aufgabenstellung und Hinweisen, die bei der Implementierung helfen können.
Die Hinweise sind \textbf{nicht bindend} und als Hilfestellung gedacht, um ein paar übliche Probleme zu vermeiden.
Eine richtige Lösung muss die Hinweise nicht zwingend umsetzen, solange sie die Aufgabenstellung erfüllt.

\subsection{Aufgabenstellung}
\begin{enumerate}
	\item Implementieren Sie eine Klasse, die die Schnittstelle \beamertxt{\linebreak}\texttt{idb.block.BlockFile} implementiert.
		Beachten Sie die Dokumentation der Methoden in der Schnittstelle.
	\item Tragen Sie den Konstruktor Ihrer Klasse in \texttt{idb.construct.Util} in der Methode \texttt{generateBlockFile(int blockSize)} ein.
		\texttt{generateBlockFile(int blockSize)} soll die Blockgröße als Integer aufnehmen und ein neue Blockfile mit dieser Blockgröße zurückgeben.
		Beachten Sie, dass die BlockFile zu diesem Zeitpunkt keine Datei im Dateisystem geöffnet hat.
	\item Sorgen Sie dafür, dass Sie alle Tests aus der Klasse \texttt{BlockTests} erfüllen.
	Sie können diese Testfälle mit \lstinline|ant Meilenstein1| ausführen.
	\item Die Abgabe auf GitLab erfolgt zeitgleich mit der Abgabe der Zusatzaufgaben des nächsten Übungsblattes auf StudOn. Markieren Sie hierfür ihre Abgabe mit dem Tag "`Aufgabe-1"'.
\end{enumerate}

\subsection{Hinweise}
\begin{itemize}
	\item Die Interaktion mit dem Dateisystem aus Java kann in unserem Fall \normaltxt{\linebreak}\texttt{java.io.RandomAccessFile} übernehmen.
	\item Ein \texttt{java.nio.ByteBuffer} ist das, was in Java am ehesten an ein \texttt{char*} aus C herankommt.
		Er übernimmt die Rolle des Speichers, in den die Daten von dem Laufwerk gelesen werden können.
	\item Für diese Aufgabe sind ausschließlich die Klassen in den Paketen \texttt{idb.block} und (zum Testen) \texttt{idb.datatypes} interessant.
		Nutzen Sie diese Aufgabe, um sich mit den Klassen in diesen Paketen vertraut zu machen.
	\item Eine nebenläufige Änderung der Dateigröße im Dateisystem muss nicht beachtet werden.
	\item Seien Sie vorsichtig mit \texttt{RandomAccessFile::read}. \texttt{read} liest nicht n bytes, sondern bis zu n bytes.
		Wenn Sie solange warten möchten, bis alle n bytes vorrätig sind, empfiehlt sich \texttt{readFully}.
	\item Testen Sie Ihre Implementierung, indem Sie den Inhalt der Klasse \texttt{idb.Main} durch Testcode für ihre BlockFile ersetzen.
		Setzen Sie \texttt{Main} auf den orginalen Zustand zurück, sobald Sie sich von der Korrektheit überzeugt haben.
\end{itemize}
