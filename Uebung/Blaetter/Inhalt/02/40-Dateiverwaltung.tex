\section{Dateiverwaltung}
Welche Operationen müssen in welcher Reihenfolge ausgeführt werden um einen Block einer (blockorientierten) Datei zu lesen?

\begin{enumerate}[\alph{enumi})]
\item Beim öffnen der Datei.
\begin{solution}
\begin{enumerate}
	\item Master File Table bzw. Dateikatalog von dem Laufwerk über die Gerätesteuerung lesen.
	Dieser Dateikatalog liegt an einer fest definierten Stelle auf dem Laufwerk.
	\item Solange die angeforderte Datei nicht gefunden wurde, gehe die Hierarchie des Dateikatalogs durch.
	Dies ist abhängig vom Dateisystem (vgl. Vorlesungsfolien~\Katalogeintraege~und SP2).
	\item Überprüfe ob die Zugriffsrechte des Benutzers mit der angeforderten Zugriffsart übereinstimmen.
	\item Lege den Dateikontrollblock für die Datei an.
\end{enumerate}
\end{solution}
\item Beim lesen aus der geöffneten Datei.
\begin{solution}
\begin{enumerate}
	\item Lese aus dem Dateikontrollblock für die Datei die Referenz zum geforderten Block.
	\item Fordere über die Gerätesteuerung den gewünschten Block an.
	\item Die Gerätesteuerung setzt die Blocknummer in den entsprechenden Zugriffsbefehl für den Hintergrundspeicher um und liefert den Block zurück.
	Bei einer HDD beinhaltet dies die Umwandlung der Blocknummer in Slot, Zylinder und Spur.
\end{enumerate}
\end{solution}
\end{enumerate}
