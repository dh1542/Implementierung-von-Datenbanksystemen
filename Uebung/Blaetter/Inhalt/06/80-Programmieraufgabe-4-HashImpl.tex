\section{Programmieraufgabe 4: HashImpl}

\subsection{Aufgabenstellung}
\begin{enumerate}
	\item Implementieren Sie eine Klasse, die die Schnittstelle \beamertxt{\linebreak}\texttt{idb.record.KeyRecordFile} implementiert.
		Beachten Sie die Dokumentation der Methoden in der Schnittstelle. Beachten Sie, dass diese Aufgabe den Übungsaufgaben zum Thema Lineares Hashing entspricht.
		Ihre Implementierung soll dieses Verhalten umsetzen.
	\item Tragen Sie den Konstruktor Ihrer Klasse in \texttt{idb.construct.Util} in der Methode \texttt{generateHash()} ein.
		Die Funktion soll den DBBuffer und zwei BlockFiles nehmen (eine zusätzliche für den Überlaufbereich) und eine \textbf{neue} HashImpl zurückgeben.
		Beachten Sie, dass die BlockFiles zu diesem Zeitpunkt schon geöffnet sind.
		Die zusätzlichen Parameter \texttt{threshhold, initCapacity} entsprechen den für das Lineare Hashing bekannten Parametern.
	\item Tragen Sie den Konstruktor Ihrer Klasse in \texttt{idb.construct.Util} in der Methode \texttt{rebuildHash()} ein.
		Die Funktion soll den DBBuffer und zwei BlockFiles nehmen und eine \textbf{existierende} HashImpl vom Laufwerk laden.
		Beachten Sie, dass die BlockFiles zu diesem Zeitpunkt schon geöffnet sind.
		Beachten Sie auch, dass Sie \texttt{threshhold, initCapacity} und eventuelle interne Zustände wieder vom Laufwerk rekonstruieren müssen.
		Anders als bei einer TID speichern wir diese hier ab, anstatt sie beim Laden neu zu berechnen.
	\item Sorgen Sie dafür, dass Sie alle Tests aus der Klasse \texttt{HashTests} erfüllen. Warnung: Diese Testfälle können 1-2 Minuten dauern.
	Sie können diese Testfälle mit \lstinline|ant Meilenstein4| ausführen.
	\item Die Abgabe auf GitLab erfolgt zeitgleich mit der Abgabe der Zusatzaufgaben des nächsten Übungsblattes auf StudOn. Markieren Sie hierfür ihre Abgabe mit dem Tag "`Aufgabe-4"'.
\end{enumerate}

\subsection{Hinweise}
\begin{itemize}
	\item Wir empfehlen, den ersten Block des Overflow-Bereiches für Metadaten zu verwenden und erst ab dem zweiten Block im Overflow-Bereich Hashing zu betreiben.
	\item \texttt{KeyRecordFile} bietet eine Methode \texttt{close()}, in der diese Metadaten serialisiert werden können.
		Es wird sichergestellt, dass nicht zwei verschiedene \mbox{HashImpl} auf den gleichen Dateien arbeiten.
	\item Im Rahmen dieser Aufgabe ist die Behandlung von Sätzen, die länger als ein (halber) Block sind, nicht notwenig. (Dies liegt daran, dass wir diese \texttt{HashImpl} nur als Sekundärindex nutzen werden.)
	\item Es empfiehlt sich, in jedem Block Platz für einen \texttt{int} freizuhalten, um dort ggf. einen Zeiger auf einen anderen Block einzutragen.
	\item Initial müssen üblicherweise neue Blöcke (in \texttt{append()} oder im Konstruktor) mindestens partiell initialisiert werden, damit nicht beliebige Daten als Zeiger o. Ä. interpretiert werden.
	\item Im Gegensatz zu TID kann man sich im \texttt{HashImpl} darauf verlassen, dass \texttt{Key.copy} jegliche Keys und \texttt{Value.copy} jegliche Values lesen und wieder serialisieren können.
	\item Achten Sie darauf, beim Einfügen und beim Split die Reihenfolgeeigenschaften in \texttt{idb.record.KeyRecordFile:insert} nicht zu verletzen.
	\item \textbf{Bonus:} Implementieren Sie eine einfache Freispeicherverwaltung für die Blöcke im Overflowbereich. Diese ist besonders einfach, da jeder Block genau gleich groß ist.
	\item Im Gegensatz zu Blöcken im Speicher können Daten die in ein \texttt{DataObject} geladen wurden problemlos nach Freigabe des Blockes verwendet werden.
	\item Achten Sie darauf, nicht zu viele Datensätze gleichzeitig aus den Blöcken in ein \texttt{DataObject} zu konvertieren. In einem echten System geht hierbei leicht der RAM aus.
	\item Folgende Hilfsmethode sind vermutlich ganz nützlich: \texttt{int hash(Key), int alloc(), void free(), List<Triplet<Key, DataObject, Integer>> listBlockContest(ByteBuffer)}
	\item Vermutlich ist es hilfreich, sich zu notieren, wie viel Platz in einem Block noch frei ist.
	\item Für \texttt{split()} ist es nötig, die Anzahl an insgesamt belegten Bytes zu kennen.
\end{itemize}
