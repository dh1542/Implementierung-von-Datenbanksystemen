\section{Höhe von B*-Bäumen}
Geben Sie für den B*-Baum eine Formel an, mit der man die obere und untere Schranke für die Höhe des Baums aus gegebenem k, $k_{Leaf}$ und der Anzahl der eingetragenen Sätze oder Satzadressen n bestimmen kann.

\begin{note}
\subsection*{Obere Schranke für B*-Bäume}
\begin{tabular}{cc}
	Tiefe & Minimale Blattknotenzahl in der Tiefe \\
	\hline
	1 & 1 \\
	\hline
	2 & $2\cdot (k+1)^0$ \\
	\hline
	3 & $2\cdot (k+1)^1$ \\
	\hline
	4 & $2\cdot(k+1)^2$ \\
	\hline
	\vdots & \vdots \\
	\hline
	h & $2(k+1)^{h-2}$ \\
	\hline
\end{tabular}

Minimale Anzahl der Elemente im Baum = $Blattknotenzahl \cdot k_{Leaf}$ (Jeder Knoten minimal gefüllt); Ausnahme: Höhe 1.

Damit gilt für die Obere Schranke:\\
Anmerkung Sonderbehandlung bei Höhe 1;
n ist zu klein für die Ebene darüber, also:
\begin{align*}
n&< \textrm{min\_Einträge}(h+1)\\
n&< 2\cdot(k+1)^{h-1}\cdot k_{Leaf} && 2\cdot k_{Leaf}>0\\
\frac{n}{2\cdot k_{Leaf}} &< (k+1)^{h-1} && \textrm{Log ist streng monoton steigend}\\
\log_{k+1} (\frac{n}{2\cdot k_{Leaf}}) &< h-1\\
\log_{k+1} (\frac{n}{2\cdot k_{Leaf}}) +1 &<h; && \textrm{für $x\in{\mathbb{R}}$ gilt:} \\
\textrm{da } h>1 \textrm{ und }h&\textrm{ maximal gewählt wurde}  && \min_{z\in\mathbb{Z}} (z) > x\Leftrightarrow x+1 \geq\max_{z\in\mathbb{Z}}(z);\\
h&\leq \log_{k+1}(\frac{n}{2\cdot k_{Leaf}}) +2
\end{align*}
Gültig nur für $h\in\mathbb{N}_2$, also
\begin{align*}
2&\leq 2+\log_{k+1}(\frac{n}{2\cdot k_{Leaf}})\\
0&\leq \log_{k+1} (\frac{n}{2\cdot k_{Leaf}}) && \textrm{Exp ist streng monoton steigend}\\
1&\leq \frac{n}{2\cdot k_{Leaf}} && 2\cdot k_{Leaf}>0\\
2\cdot k_{Leaf}&\leq n
\end{align*}
Somit benötigen wir für $n<2\cdot k_{Leaf}$ eine extra Funktion:\\
$h_{oben}(n):=1$ da wir hier immer nur einen Blattknoten haben.\\
Insgesamt ergibt dies:\\
\begin{align*}
h_{oben}(n): =  &1& &\textrm{wenn }n<2\cdot k_{Leaf}\\
 &\log_{k+1}(\frac{n}{2\cdot k_{Leaf}})+2& & sonst
\end{align*}
\subsection*{Untere Schranke}
\begin{tabular}{cc}
	Tiefe & Maximale Blattknotenzahl in der Tiefe \\
	\hline
	1 & 1 \\
	\hline
	2 & $(2k+1)$ \\
	\hline
	3 & $(2k+1)^2$ \\
	\hline
	4 & $(2k+1)^3$ \\
	\hline
	\vdots & \vdots \\
	\hline
	h & $(2k+1)^{h-1}$ \\
	\hline
\end{tabular}

Maximale Anzahl der Elemente im Baum = $Blattknotenzahl \cdot 2\cdot k_{Leaf}$ (Jeder Knoten maximal gefüllt);

Damit gilt für die Untere Schranke:\\
Anmerkung: Sonderbehandlung bei Höhe 1;\\
n ist zu groß für die Ebene darunter, also:
\begin{align*}
n&> \textrm{max\_Einträge}(h-1)&& \textrm{Fordert wieder Sonderbehandlung von Höhe h=1}\\
n&> (2k+1)^{h-2}\cdot 2\cdot k_{leaf}&& 2*k_{leaf} >0\\
\frac{n}{2\cdot  k_{Leaf}} &> (2k+1)^{h-2} && \textrm{Log ist streng monoton steigend}\\
\log_{2k+1} (\frac{2}{2\cdot k_{Leaf}}) &> h-2\\
\log_{2k+1} (\frac{2}{2\cdot k_{Leaf}}) +2 &>h; && \textrm{für $x\in{\mathbb{R}}$ gilt:} \\
\textrm{da } h>1 \textrm{ und }h&\textrm{ minimal gewählt wurde}  && \min_{z\in\mathbb{Z}} (z) \geq x\Leftrightarrow x+1 >\max_{z\in\mathbb{Z}}(z);\\
h&\geq 1+\log_{2k+1}(\frac{n}{2\cdot k_{Leaf}})
\end{align*}
Gültig wieder nur für $h\in\mathbb{N}_2$

Für $n<2\cdot k_{Leaf}$ ist die Schranke wieder durch $h_{unten}(n):=1$ gegeben.\\
Insgesamt ergibt dies:\\
\begin{align*}
h_{unten}(n): = & 1 &&\textrm{wenn }n<2\cdot k_{Leaf}\\
& 1+\log_{2k+1}(\frac{n}{2\cdot k_{Leaf}}) && sonst
\end{align*}
\end{note}
