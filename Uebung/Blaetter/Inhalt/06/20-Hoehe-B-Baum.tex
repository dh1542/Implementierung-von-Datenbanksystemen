\section{Höhe von B-Bäumen}
Geben Sie für den B-Baum eine Formel an, mit der man die obere und untere Schranke für die Höhe des Baums aus gegebenem $k$ und der Anzahl der eingetragenen Sätze oder Satzadressen $n$ bestimmen kann ($n\in\mathbb{N}_1$).

\begin{note}
	Siehe auch \url{https://vsis-www.informatik.uni-hamburg.de/oldServer/teaching/ws-12.13/gdb/folien/Haerder-Baeume-half.pdf\#page=5}
\end{note}

\begin{solution}

\subsection*{Obere Schranke für die Höhe des B-Baums}
\begin{tabular}{p{2cm} p{8cm}}
	Tiefe			& Minimale Knotenzahl in der Tiefe		\\
	\hline
	1				& 1 														\\
	\hline
	2 				& $2(k+1)^0$ 										\\
	\hline
	3 				& $2(k+1)^1$  									\\
	\hline
	4 				& $2(k+1)^2$ 										\\
	\hline
	\ldots 		& \ldots												\\
	\hline
	h 				& $2(k+1)^{h-2}$ 								\\
	\hline
\end{tabular}

Definition geometrische Reihe:
\begin{align*}
\sum_{i=0}^n q^i &= \frac{q^{n+1}-1}{q-1}
\end{align*}
Minimale Knotenzahl im Baum bei gegebener Höhe =
Summe über Minimalzahl der einzelnen Ebenen

\begin{align*}
1 + \sum_{i = 2}^h 2\cdot (k + 1)^{i-2} = 1 + \sum_{i  = 0}^{h-2} 2 \cdot (k +1)^{i}
\overset{\text{geometr.}}{\underset{\text{Reihe}}{=}} 1 + 2 \cdot \frac{(k +1)^{h-1} - 1}{k + 1 - 1}
= 1 + 2 \cdot \frac{(k + 1)^{h - 1} - 1}{k}
\end{align*}

Minimale Anzahl der Elemente (TIDs) im Baum =
$\mathrm{Knotenzahl} \cdot k$ (jeder Knoten minimal gefüllt)
\begin{align*}
1 + k \cdot 2 \frac{(k + 1)^{h - 1} - 1}{k} = 1 + 2 \cdot (k + 1)^{h - 1} - 2 = 2 \cdot (k + 1)^{h - 1} - 1
\end{align*}
Wurzel kann auch weniger als $k$ Elemente beinhalten.

Es gilt also:
\begin{align*}
2 \cdot (k + 1)^{h - 1} - 1 &\leq n\\
2 \cdot (k + 1)^{h - 1} &\leq n + 1\\
(k + 1)^{h - 1} &\leq \frac{n + 1}{2}\\
h - 1 &\leq \log_{k +1} \text{\huge{(}}\frac{n + 1}{2}\text{\huge{)}}\\
h &\leq \log_{k +1} \text{\huge{(}}\frac{n + 1}{2}\text{\huge{)}} + 1
\end{align*}

Damit erhalten wir:
\begin{align*}
h_{oben}(n):= & \log_{k +1} \text{\huge{(}}\frac{n + 1}{2}\text{\huge{)}} + 1
\end{align*}


\subsection*{Untere Schranke für die Höhe des B-Baums}
\begin{tabular}{p{2cm} p{8cm}}
	Tiefe			& Maximale Knotenzahl in der Tiefe	\\
	\hline
	1				& 1 = $(2k + 1)^0$ 							\\
	\hline
	2 				& $(2k+1)^1$ 										\\
	\hline
	3 				& $(2k+1)^2$ 										\\
	\hline
	4 				& $(2k+1)^3$ 										\\
	\hline
	\ldots		& \ldots												\\
	\hline
	h 				& $(2k+1)^{h-1}$ 								\\
	\hline
\end{tabular}

Maximale Knotenzahl im Baum =
Summe über Maxima je Ebene
\begin{align*}
\sum_{i = 1}^{h} (2k + 1)^{i - 1} = \sum_{i = 0}^{h - 1} (2k + 1)^{i} \overset{\text{geometr.}}{\underset{\text{Reihe}}{=}} \frac{(2k + 1)^h - 1}{2k + 1 - 1} = \frac{(2k + 1)^h - 1}{2k}
\end{align*}
Maximale Anzahl der Elemente(TIDs) im Baum =
$\mathrm{Knotenzahl} \cdot 2k$ (jeder Knoten voll gefüllt)

\begin{align*}
2k \cdot \frac{(2k + 1)^h - 1}{2k} = (2k + 1)^h - 1
\end{align*}
Es gilt also:
\begin{align*}
(2k + 1)^h - 1 &\geq n\\
(2k + 1)^h &\geq n + 1\\
h & \geq \log_{2k + 1} (n +1)
\end{align*}
Damit erhalten wir:
\begin{align*}
h_{unten}(n):=& \log_{2k + 1} (n +1)
\end{align*}
% Ein Comment darf andscheinend eine maximale länge nicht überschreiten, daher zwei...
\end{solution}
\begin{solution}
\subsection*{Was bedeutet das nun?}
Beispiel:
\begin{itemize}
	\item Blockgröße: 4\,kiB
	\item Schlüssellänge: 10\,B
	\item Blockzeiger: 4\,B (erlaubt bei dieser Blockgröße Files bis ca. 16\,TiB)
	\item TID: 5\,B (Blockzeiger + 1\,B für die Satznummer)
	\item Satz: 117\,B
\end{itemize}
Daraus ergeben sich für k:

\begin{tabular}{p{4cm} p{2cm} p{3.5cm} p{3.5cm}}
										& B-Baum 	& B*-Baum Knoten	& B*-Baum Blatt	\\
	\hline
	Primärorganisation 		& 15 			& 146 						& 16 					\\
	\hline
	Sekundärorganisation 	& 107 			& 146 						& 136					\\
	\hline
\end{tabular}

Rechenweg:
\begin{align*}
\text{Größe B-Baum Knoten} &= 
((2k + 1) \cdot \text{Blockzeiger} + 2k \cdot \text{Schlüssel} + 2k \cdot \text{(TID oder Satz)})\\
&= 2k \cdot (\text{Blockzeiger} + \text{Schlüssel} + \text{(TID oder Satz)}) + \text{Blockzeiger}\\
&= 2k \cdot (4 + 10 + (5\text{ oder }117)) + 4 \text{[Byte]} = 2k \cdot (19\text{ oder }131) + 4 \text{[Byte]}\\
\text{TID}_{k = 107} &=2 \cdot 107 \cdot 19 + 4 = 4070\\
\text{TID}_{k = 108} &=2 \cdot 108 \cdot 19 + 4 = 4108\\
\text{Satz}_{k = 15} &= 2 \cdot 15 \cdot 131 + 4 = 3934\\
\text{Satz}_{k = 16} &= 2 \cdot 16 \cdot 131 + 4 = 4196\\
\text{B*-Knoten}_{k=146} &= 2\cdot 146 \cdot (4 + 10) + 4 = 4092\\
\text{B*-Blatt-TID}_{k=136} &= 2\cdot 136\cdot (10+5) = 4080\\
\text{B*-Blatt-Satz}_{k=16} &= 2\cdot 16\cdot (10+117) = 4064\\
\text{B*-Blatt-Satz}_{k=17} &= 2\cdot 17\cdot (10+117) = 4318\\
\end{align*}

\begin{note}
Die Werte alle zu bestimmen dauert 10-15 Minuten! Danach können wir kein Beispiel rechnen. $\rightarrow$ Werte angeben und dann die Höhe bestimmen.
\end{note}

Für 1.000.000 Datensätze ergibt sich:

\begin{tabular}{p{4cm} p{3cm} p{3cm}}
	B-Baum-Höhe				& min 						& max						\\
	\hline
	Primärorganisation 		& $4,023169201$	& $5,732892503$ 	\\
	\hline
	Sekundärorganisation 	& $2,572415323$	& $3,802647713$	\\
	\hline
\end{tabular}

Erkenntnis: Als Sekundärorganisation hat der B-Baum häufig eine Tiefe von 3, bei 100.000.000 immer noch nur 4.
Der B*-Baum ist noch etwas flacher (insbesondere als Primärorganisation), aber das betrachten wir nicht in dieser Übung.
\end{solution}
