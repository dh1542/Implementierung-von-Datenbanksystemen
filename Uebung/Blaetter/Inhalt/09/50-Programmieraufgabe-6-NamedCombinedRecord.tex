\section{Programmieraufgabe 6: NamedCombinedRecord}

\subsection{Aufgabenstellung}
\begin{enumerate}
	\item Implementieren Sie eine Klasse, die die Schnittstelle \beamertxt{\linebreak}\texttt{idb.datatypes.NamedCombinedRecord} implementiert.
		Beachten Sie die Dokumentation der Methoden in der Schnittstelle.
		Ein \texttt{NamedCombinedRecord} entspricht einem Tupel in klassischer Speicherungsart (kein C-Store) als ein Satz.
		Wir behandeln ausschließlich die drei Typen \texttt{BOOL, STRING} und \texttt{INT}, die mit den Klasse \texttt{Bool, DBString} und \texttt{IntegerKey} aus \texttt{idb.datatypes} zusammengehören.
		Ein \texttt{NamedCombinedRecord} besteht aus drei Teilen: Einer Datentypstruktur (z.B. erst zwei \texttt{INT}, dann zwei \texttt{STRING}, dann wieder ein \texttt{INT} und abschließend einen \texttt{BOOL}), einer Benennung (die erste Spalte heißt \texttt{Alter}, die zweite \texttt{Note}, die dritte \texttt{Wohnort}, ...),
		und den Werten für diese Spalten (das Alter ist \texttt{21}, die Note \texttt{50}, der Wohnort \texttt{Erlangen}, ...)

		Wie jedes \texttt{DataObject} bietet auch der \texttt{NamedCombinedRecord} die Möglichkeit der Serialisierung mit \texttt{write} und \texttt{read}.
		Dabei sollen nur die Werte, nicht aber die Struktur oder die Namen abgespeichert werden.
		Beim Auslesen von abgespeicherten Sätzen darf als sicher angenommen werden, dass die Struktur korrekt ist.
	\item Tragen Sie den Konstruktor Ihrer Klasse in \texttt{idb.construct.Util} in der Methoden \texttt{namedCombinedRecordFrom()} ein.
		Diese Methoden nehmen einen Namen und einen Wert (einen \texttt{String}, eine \texttt{int} oder einen \texttt{boolean}) und sollen einen \texttt{NamedCombinedRecord} zurückgeben, der ausschließlich aus diesem einem Typen und dem mitgegebenen Wert besteht.
	\item Tragen Sie außerdem einen Kontruktor Ihrer Klasse in \texttt{idb.construct.Util} in die Methode \texttt{getNCR()} ein.
		Diese Methode nimmt ausschließlich eine Liste, Datentypen und Namen und erhält keine definierten Werte. Sie dürfen diese auf beliebige Werte initialisieren.
	\item Sorgen Sie dafür, dass Sie alle Tests aus der Klasse \texttt{NamedCombinedRecordTests} erfüllen.
	Sie können diese Testfälle mit \lstinline|ant Meilenstein6| ausführen.
	\item Die Abgabe auf GitLab erfolgt zeitgleich mit der Abgabe der Zusatzaufgaben des nächsten Übungsblattes auf StudOn. Markieren Sie hierfür Ihre Abgabe mit dem Tag "`Aufgabe-6"'.
\end{enumerate}

\subsection{Hinweise}
\begin{itemize}
	\item Achten Sie bei \texttt{combine, modifyValues, get} und \texttt{rename} darauf, die Struktur beizubehalten.
	\item Bis auf \texttt{read / write} ist ein \texttt{NamedCombinedRecord} nicht veränderlich, stattdessen geben \texttt{combine, modifyValues, get, rename} Kopien mit entspechenden Anpassungen heraus. Achten Sie darauf, die Kopie vollständig durchzuführen (\texttt{DataObject::copy})
	\item Eine Speicherung der Längen, wie es in Vorlesung und Übung betrachtet wird, ist in diesem Fall nicht nötig, da diese bereits in den zu verwendenden Klassen \texttt{DBString, Bool, IntegerKey} gehandhabt wird (falls relevant).
	\item In \texttt{read / write} können Sie davon ausgehen, dass der Satz nicht fragmentiert ist.
		Eine Fragmentierung würde anders (von \texttt{TID-File} oder \texttt{SeqRecordFile}) behandelt werden, sodass dies für den Satz transparent (= unsichtbar) passiert.
\end{itemize}
