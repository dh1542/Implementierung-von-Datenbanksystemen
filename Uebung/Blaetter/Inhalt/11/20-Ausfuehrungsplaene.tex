\section{Ausführungspläne}

Gegeben seien folgende Relationen:

\texttt{Employee (\underline{num}, fname, lname, addr, mgr[employee],\beamertxt{\linebreak} dep[department])}

\texttt{Department (\underline{num}, name, mgr[employee])}

\texttt{Project (\underline{num}, dep[department], mgr[employee])}

\texttt{Works\_on (\underline{proj[project], empl[employee]})}

Darüber hinaus wurden folgende Sichten erstellt:

\texttt{
    CREATE VIEW EmployeesManager AS \\
    \hspace*{.3cm}  SELECT e.lname as empl\_lname, s.lname as mgr\_lname \\
    \hspace*{.3cm}  FROM employee e, employee s \\
    \hspace*{.3cm}  WHERE e.mgr = s.num
    }

\texttt{
    CREATE MATERIALIZED VIEW ResearchMembers AS \\
    \hspace*{.3cm}  SELECT e.fname as fname, e.lname as lname,\beamertxt{\linebreak\hspace*{.cm}} e.addr as addr \\
    \hspace*{.3cm}  FROM employee e \\
    \hspace*{.3cm}  JOIN department d \\
    \hspace*{.3cm}  ON d.num = e.dep \\
    \hspace*{.3cm}  WHERE d.name = 'Research'
    }

\begin{note}
Die nicht-optimierten Lösungen entsprechen zum Teil denen von Blatt 10.
Sie müssen dementsprechend nicht zwangsläufig erneut hergeleitet werden.
\end{note}

Setzen Sie folgende SQL-Anweisungen zunächst in nicht-optimierte Operatorgraphen um. Verwenden Sie dafür die Baum-Notation von Vorlesungsfolie~\Operatorgraph.

Optimieren Sie anschließend die Operatorgraphen wie von Blatt 10 gewohnt,
nun aber auch unter Berücksichtigung der oben definierten Sichten.

\begin{enumerate}[a)]

	\item
Alle MitarbeiterInnen und ihre Vorgesetzten
	\begin{lstlisting}
SELECT empl_lname, mgr_lname
FROM   EmployeesManager e;
	\end{lstlisting}

\begin{solution}
Die Anfrage mit der Sicht sieht wie folgt aus:

	\begin{tikzpicture}
		\node (proj) at (0, 0) {\texttt{PROJ(\qquad, (empl\_lname, mgr\_lname))} };
		\node (ren)[below of =proj, xshift =-3cm] {\texttt{RENAME(\qquad, e)}};
		\node (emp)[below of =ren, xshift =2cm] {\texttt{EmployeesManager} };

		\draw[-triangle 60] (ren) -- ($(proj.south) + (-2.2,0.1)$);
		\draw[-triangle 60] (emp) -- ($(ren.south) + (0.1,0.1)$);
	\end{tikzpicture}

Da es sich aber nicht um eine materialisierte Sicht handelt,
muss der Operatorbaum der Sicht selbst aufgestellt und eingesetzt werden:

	\begin{tikzpicture}
		\node (rename) at (0,0) {\texttt{RENAME(\hspace{0.5cm}, (e.lname AS empl\_lname, s.lname AS mgr\_lname))}};
		\node (proj) [below of = rename, xshift=-2.8cm] {\texttt{PROJ(\hspace{0.5cm} , (e.lname, s.lname))} };
		\node (sel) [below of =proj] {\texttt{SEL(\hspace{0.5cm} , (e.mgr = s.num))} };
		\node (cross)[below of =sel] {\texttt{CROSS(\qquad, \qquad)} };
		\node (ren2)[below of =cross, xshift = 2cm] {\texttt{RENAME(\qquad, s)}};
		\node (ren1)[below of =cross, xshift =-3cm] {\texttt{RENAME(\qquad, e)}};
		\node (emp1) [below of =ren2] {\texttt{employee}};
		\node (emp2) [below of =ren1] {\texttt{employee}};

		\draw[-triangle 60] (proj) -- ($(rename.south) + (-4.2,0)$);
		\draw[-triangle 60] (sel) -- ($(proj.south) + (-1.4,0)$);
		\draw[-triangle 60] (cross) -- ($(sel.south) + (-1.4,0)$);
		\draw[-triangle 60] (ren1) -- ($(cross.south) + (0.0,0.1)$);
		\draw[-triangle 60] (ren2) -- ($(cross.south) + (0.8,0.1)$);
		\draw[-triangle 60] (emp1) -- ($(ren2.south) + (0.2,0.1)$);
		\draw[-triangle 60] (emp2) -- ($(ren1.south) + (0.5,0.1)$);
	\end{tikzpicture}

Setzt man nun den Baum der Sicht ein in den Baum der Anfrage, ergibt sich:

	\begin{tikzpicture}
        \node (proj2) at (0, 0) {\sout{\texttt{PROJ(\qquad, (empl\_lname, mgr\_lname))}}};
		\node (ren)[below of =proj2] {\sout{\texttt{RENAME(\qquad, e)}}};
		\node (rename)[below of =ren] {\texttt{RENAME(\hspace{0.5cm}, (e.lname AS empl\_lname, s.lname AS mgr\_lname))}};
		\node (proj) [below of = rename, xshift=-2.8cm] {\texttt{PROJ(\hspace{0.5cm} , (e.lname, s.lname))} };
		\node (sel) [below of =proj] {\texttt{SEL(\hspace{0.5cm} , (e.mgr = s.num))} };
		\node (cross)[below of =sel] {\texttt{CROSS(\qquad, \qquad)} };
		\node (ren2)[below of =cross, xshift = 2cm] {\texttt{RENAME(\qquad, s)}};
		\node (ren1)[below of =cross, xshift =-3cm] {\texttt{RENAME(\qquad, e)}};
		\node (emp1) [below of =ren2] {\texttt{employee}};
		\node (emp2) [below of =ren1] {\texttt{employee}};

		\draw[-triangle 60] (ren) -- ($(proj2.south) + (-2,0)$);
		\draw[-triangle 60] (rename) -- ($(ren.south) + (0.3,0)$);
		\draw[-triangle 60] (proj) -- ($(rename.south) + (-4.2,0)$);
		\draw[-triangle 60] (sel) -- ($(proj.south) + (-1.4,0)$);
		\draw[-triangle 60] (cross) -- ($(sel.south) + (-1.4,0)$);
		\draw[-triangle 60] (ren1) -- ($(cross.south) + (0.0,0.1)$);
		\draw[-triangle 60] (ren2) -- ($(cross.south) + (0.8,0.1)$);
		\draw[-triangle 60] (emp1) -- ($(ren2.south) + (0.2,0.1)$);
		\draw[-triangle 60] (emp2) -- ($(ren1.south) + (0.5,0.1)$);
	\end{tikzpicture}

Die oberste Projektion ist offensichtlich redundant und kann wegoptimiert werden.
\end{solution}

	\item
Alle MitarbeiterInnen der Forschungsabt. in der Martensstr. 3

\cprotEnv
\begin{normalText}
\begin{lstlisting}[showstringspaces=false]
SELECT e.fname, e.lname, e.addr
FROM   employee e
JOIN   department d
ON     d.num = e.dep
WHERE  d.name = 'Research' AND e.addr = 'Martensstr. 3';
\end{lstlisting}
\end{normalText}

\cprotEnv
\begin{beamerText}
\begin{lstlisting}[showstringspaces=false]
SELECT e.fname, e.lname, e.addr
FROM   employee e
JOIN   department d
ON     d.num = e.dep
WHERE  d.name = 'Research'
  AND e.addr = 'Martensstr. 3';
\end{lstlisting}
\end{beamerText}
%\end{beamerText}

\begin{solution}
Der nicht-optimierte Baum ist weitgehend von Blatt 10 bekannt:

  \begin{tikzpicture}
		\node (proj) at (0, 0) {\texttt{PROJ(\qquad , (e.fname, e.lname, e.addr))} };
		\node (sel)[below of =proj,xshift=-1cm] {\texttt{SEL(\qquad , d.name = 'Research' AND e.addr = 'Martensstr. 3'))} };
		\node (join)[below of =sel,xshift=-2cm] {\texttt{JOIN(\qquad, \qquad, (d.num = e.dep))} };
		\node (ren2)[below of =join, xshift = 2cm] {\texttt{RENAME(\qquad, d)}};
		\node (ren1)[below of =join, xshift =-3cm] {\texttt{RENAME(\qquad, e)}};
		\node (dept)[below of =ren2] {\texttt{department} };
		\node (emp)[below of =ren1] {\texttt{employee} };

		\draw[-triangle 60] (dept) -- ($(ren2.south) + (0.4,0.1)$);
		\draw[-triangle 60] (emp) -- ($(ren1.south) + (0.5,0.1)$);
		\draw[-triangle 60] (ren1) -- ($(join.south) + (-2,0.1)$);
		\draw[-triangle 60] (ren2) -- ($(join.south) + (-1,0.1)$);
		\draw[-triangle 60] (join) -- ($(sel.south) + (-5,0.1)$);
		\draw[-triangle 60] (sel) -- ($(proj.south) + (-2.6,0.1)$);
	\end{tikzpicture}

Das Besondere ist nun, dass eine materialisierte Sicht zur Verfügung steht,
die vom System genutzt werden kann, um die Anfrage zu optimieren.

Der Operatorbaum der Sicht sieht so aus:

  \begin{tikzpicture}
		\node (proj) at (0, 0) {\texttt{PROJ(\qquad , (e.fname, e.lname, e.addr))} };
		\node (sel)[below of =proj] {\texttt{SEL(\qquad , d.name = 'Research'))} };
		\node (join)[below of =sel] {\texttt{JOIN(\qquad, \qquad, (d.num = e.dep))} };
		\node (ren2)[below of =join, xshift = 2cm] {\texttt{RENAME(\qquad, d)}};
		\node (ren1)[below of =join, xshift =-3cm] {\texttt{RENAME(\qquad, e)}};
		\node (dept)[below of =ren2] {\texttt{department} };
		\node (emp)[below of =ren1] {\texttt{employee} };

		\draw[-triangle 60] (dept) -- ($(ren2.south) + (0.4,0.1)$);
		\draw[-triangle 60] (emp) -- ($(ren1.south) + (0.5,0.1)$);
		\draw[-triangle 60] (ren1) -- ($(join.south) + (-2,0.1)$);
		\draw[-triangle 60] (ren2) -- ($(join.south) + (-1,0.1)$);
		\draw[-triangle 60] (join) -- ($(sel.south) + (-1.8,0.1)$);
		\draw[-triangle 60] (sel) -- ($(proj.south) + (-2.6,0.1)$);
	\end{tikzpicture}

Da die Daten der materialisierten Sicht bereits physisch vorliegen,
kann sie wie jede andere Relation im Operatorbaum verwendet werden.
So kann auf die tatsächliche Ausführung der evtl.\ kostenintensiven JOIN-Operationen
und der damit verbundenen Table-Scans verzichtet werden, indem die bereits
von der materialisierten Sicht vorgehaltenen Daten verwendet werden.

Um nun einen Teilbaum des ursprünglichen Operatorbaums der Anfrage zu ersetzen,
muss die Selektion noch aufgeteilt werden und die Projektion vorgezogen werden:

  \begin{tikzpicture}
		\node (sel2) at (0, 0) {\texttt{SEL(\qquad , e.addr = 'Martensstr. 3'))} };
		\node (proj)[below of =sel2] {\texttt{PROJ(\qquad , (e.fname, e.lname, e.addr))} };
		\node (sel)[below of =proj] {\texttt{SEL(\qquad , d.name = 'Research'))} };
		\node (join)[below of =sel] {\texttt{JOIN(\qquad, \qquad, (d.num = e.dep))} };
		\node (ren1)[below of =join, xshift = 2cm] {\texttt{RENAME(\qquad, d)}};
		\node (ren2)[below of =join, xshift =-3cm] {\texttt{RENAME(\qquad, e)}};
		\node (dept)[below of =ren2] {\texttt{department} };
		\node (emp)[below of =ren1] {\texttt{employee} };

		\draw[-triangle 60] (dept) -- ($(ren2.south) + (0.2,0.1)$);
		\draw[-triangle 60] (emp) -- ($(ren1.south) + (0.2,0.1)$);
		\draw[-triangle 60] (join) -- ($(sel.south) + (-1.8,0.1)$);
		\draw[-triangle 60] (ren2) -- ($(join.south) + (-2.0,0.1)$);
		\draw[-triangle 60] (ren1) -- ($(join.south) + (-0.8,0.1)$);
		\draw[-triangle 60] (sel) -- ($(proj.south) + (-2.5,0.1)$);
		\draw[-triangle 60] (proj) -- ($(sel2.south) + (-2.5,0.1)$);
	\end{tikzpicture}

Nun kann der untere Teil des Operatorbaums komplett ersetzt werden:

  \begin{tikzpicture}
    \node (sel) at (0, 0) {\texttt{SEL(\qquad , e.addr = 'Martensstr. 3'))} };
		\node (ren)[below of =sel] {\texttt{RENAME(\qquad , e)} };
		\node (mem)[below of =ren] {\texttt{ResearchMembers} };

		\draw[-triangle 60] (ren) -- ($(sel.south) + (-2.4,0.1)$);
		\draw[-triangle 60] (mem) -- ($(ren.south) + (0.3,0.1)$);
	\end{tikzpicture}

\paragraph{\color{solutioncolor}Zusatzfrage:} Unter welchen Bedingungen können materialisierte Sichten
automatisch zur Optimierung genutzt werden, obwohl sie in der formulierten
Anfrage nicht vorkommen\footnote{Nach: Hector Garcia-Molina, Jeffrey D. Ullman, Jennifer Widom: \emph{Database Systems -- The Complete Book}. 2nd edition, Pearson, 2013, ISBN 1-292-02447-X, Kap.~8}?

Angenommen, es gibt eine materialisierte Sicht V, die durch eine Anfrage der
folgenden Form definiert ist:

  \texttt{SELECT $L_V$ FROM $R_V$ WHERE $C_V$}

Dabei ist $L_V$ eine Liste von Attributen, $R_V$ eine Liste von Relationen
und $C_V$ eine Bedingung.

Außerdem haben wir eine Anfrage Q der Form:

  \texttt{SELECT $L_Q$ FROM $R_Q$ WHERE $C_Q$}

Dann kann Q (zumindest teilweise) umgeschrieben werden, um V zu nutzen, wenn gilt:

  \begin{itemize}
    \item Alle Relationen in $R_V$ kommen auch in $R_Q$ vor.
    \item $C_Q$ ist äquivalent zu \texttt{$C_V$ AND $C$}.
      (Wobei $C_V$ in der umgeschriebenen Anfrage dann verschwindet bzw.
      in der Sicht bereits enthalten ist und $C$ die restliche Bedingung
      aus $C_Q$ ist, die von $C_V$ noch nicht abgedeckt ist.)
    \item Die Attribute aus $L_Q$, die sich auf Relationen aus $R_V$
      beziehen, müssen ebenfalls in $L_V$ vorkommen.
    \item Falls $C$ nicht leer ist,
       müssen auch die in $C$ verwendeten Attribute,
       die sich auf Relationen aus $R_V$ beziehen, in $L_V$ vorkommen.
    \end{itemize}

\paragraph{\color{solutioncolor}Anmerkung:}
Die erste Bedingung wirkt auf den ersten Blick vielleicht unnötig,
da zusätzliche Informationen,
die aus einer Relation in $R_V$ stammen und
nicht in der Ausgabe enthalten sein sollen,
durch die zweite Bedingung und die Projektion wieder eliminiert würden.
Das stimmt zwar für die Informationen an sich,
ein Problem stellt allerdings die Multimengen-Eigenschaft dar.

Beispiel: Es existieren zwei Relationen
\texttt{A(a1, a2)} und \texttt{B(b1, b2)}.
Außerdem gibt es eine materialisierte Sicht,
die das Kreuzprodukt aus A und B bildet.
Nun wird eine Anfrage $Q$ ausgeführt, die sich nur auf \texttt{B} bezieht.
Angenommen, die Sicht wird zur Ausführung von $Q$ herangezogen.
Die Projektion würde zwar die Attribute \texttt{A.a1} und
\texttt{A.a2} eliminieren,
da aber normalerweise keine Duplikateliminierung durchgeführt wird,
wäre jedes Tupel aus \texttt{B} $n$-mal in der Ausgabe enthalten,
wobei $n$ die Anzahl der Tupel in \texttt{A} ist.

\paragraph{\color{solutioncolor}Weitere Anmerkung:}
Diese Aufgabe ist nur ein Spezialfall.
Auch komplexere Anfragen können umgeschrieben werden,
dann aber nicht unbedingt mit so einfachen Regeln.
\end{solution}

\end{enumerate}
