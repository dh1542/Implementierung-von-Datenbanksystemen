\section{Unnesting 2}
\label{unnesting2}

Gegeben seien folgende Relationen eines Datenbankschemas, die aus Aufgabe \ref{unnesting} bekannt sind:

\texttt{Studierende (\underline{matrikelnr}, name)} \\
\texttt{Noten (\underline{pruefungsnr, matrikelnr}, note, semester, datum)}

Bauen Sie die folgenden SQL-Anfragen so um, dass keine korrelierten Unteranfragen mehr vorliegen.

\cprotEnv
\begin{normalText}
Hinweis: Es ist sinnvoll, die Teilaufgaben in der gegebenen Reihenfolge zu bearbeiten, da jede Teilaufgabe eine neue Problemstellung zur vorhergehenden Teilaufgabe hinzufügt.

\begin{enumerate}[a)]

\item

\begin{lstlisting}
-- Die Matrikelnummern der besten Studierenden jedes Pruefungstags
SELECT n.matrikelnr, n.datum
FROM   Noten n
WHERE  n.note = (SELECT min (n2.note)
                 FROM   Noten n2
                 WHERE  n.datum = n2.datum);
\end{lstlisting} \label{first}

\cprotEnv
\begin{note}
\begin{lstlisting}
SELECT n.matrikelnr, n.datum
FROM   Noten n,
(
 SELECT   min(n2.note) as min, n2.datum
 FROM     Noten n2
 GROUP BY n2.datum
) m
WHERE n.note = m.min AND n.datum = m.datum
\end{lstlisting}
\end{note}

\item

\begin{lstlisting}
-- Zu jeder Pruefungsteilnahme die bis zu dem Datum
-- besten Noten des Pruefungsteilnehmers
SELECT n.matrikelnr, n.datum, n2.note, n2.pruefungsnr
FROM   Noten n, Noten n2
WHERE  n.matrikelnr = n2.matrikelnr
AND    n2.datum < n.datum
AND    n2.note = (SELECT min(n3.note)
                  FROM   Noten n3
                  WHERE  n3.datum < n.datum
		  AND    n3.matrikelnr = n.matrikelnr)
\end{lstlisting}

\cprotEnv
\begin{note}
\begin{lstlisting}
SELECT   n.matrikelnr, n.datum, n2.note, n2.pruefungsnr
FROM     Noten n, Noten n2, Noten n3
WHERE    n2.matrikelnr = n.matrikelnr
AND      n3.matrikelnr = n.matrikelnr
AND      n2.datum  < n.datum
AND      n3.datum < n.datum
GROUP BY n.matrikelnr, n.datum, n2.note, n2.pruefungsr
HAVING   n2.note = min(n3.note)
\end{lstlisting}

Alternativ, näher an der ML:

\begin{lstlisting}
SELECT   n.matrikelnr, n.datum, n2.note, n2.pruefungsnr
FROM     Noten n, Noten n2,
(
 SELECT   min(n3.note) as min, n3.datum, n3.matrikelnr
 FROM     Noten n3
 GROUP BY n3.datum, n3.matrikelnr
) m
WHERE    n.matrikelnr = n2.matrikelnr
AND      m.matrikelnr = n.matrikelnr
AND      m.datum  < n.datum
AND      n2.datum < n.datum
GROUP BY n.matrikelnr, n.datum, n2.note, n2.pruefungsr
HAVING   n2.note = min(m.min)
\end{lstlisting}
\end{note}

\item

\begin{lstlisting}
-- Fuer eine lustige Lotterie werden zu jeder Pruefung
-- die Studierenden gesucht, deren Anzahl an
-- Pruefungsteilnahmen vor dem Tag der Pruefung der
-- Pruefungsnummer entspricht
SELECT n.pruefungsnr, n.matrikelnr
FROM Noten n
WHERE n.pruefungsnr = (SELECT count(*)
                       FROM   Noten n2
                       WHERE  n2.matrikelnr = n.matrikelnr
                       AND    n2.datum < n.datum)
\end{lstlisting}

\cprotEnv
\begin{note}
\begin{lstlisting}
SELECT   n.matrikelnr, n.pruefungsnr
FROM     Noten n, Noten n2
WHERE    n2.matrikelnr = n.matrikelnr
AND      n2.datum < n.datum
GROUP BY n.matrikelnr, n.pruefungsnr
HAVING   n.pruefungsnr = count(*)
\end{lstlisting}

Alternativ, näher an der ML:

\begin{lstlisting}
SELECT   n.matrikelnr, n.pruefungsnr
FROM     Noten n,
(
SELECT   count(n2.note) as count, n2.matrikelnr, n2.datum
FROM     Noten n2
GROUP BY n2.datum, n2.matrikelnr
) m
WHERE    n.matrikelnr = m.matrikelnr
AND      m.datum < n.datum
GROUP BY n.matrikelnr, n.pruefungsnr
HAVING   n.pruefungsnr = sum(m.count)
\end{lstlisting}
\end{note}

\end{enumerate}
\end{normalText}
