\section{Unnesting}
\label{unnesting}

Gegeben seien folgende Relationen eines Datenbankschemas:

\texttt{Studierende (\underline{matrikelnr}, name)} \\
\texttt{Noten (\underline{pruefungsnr, matrikelnr}, note, semester, datum)}

Eine Anfrage, die für jede(n) Studierende(n) die beste Prüfungsleistung ermittelt,
könnte wie folgt aussehen:

\cprotEnv
\begin{normalText}
\begin{lstlisting}
SELECT s.name, n.pruefungsnr, n.note
FROM   Studierende s, Noten n
WHERE  s.matrikelnr = n.matrikelnr AND
       n.note = (SELECT min(n2.note)
                 FROM Noten n2
                 WHERE s.matrikelnr = n2.matrikelnr);
\end{lstlisting}
\end{normalText}

\cprotEnv
\begin{beamerText}
\begin{lstlisting}
SELECT s.name, n.pruefungsnr, n.note
FROM   Studierende s, Noten n
WHERE  s.matrikelnr = n.matrikelnr AND
    n.note = (SELECT min(n2.note)
              FROM Noten n2
              WHERE s.matrikelnr = n2.matrikelnr);
\end{lstlisting}
\end{beamerText}


\begin{note}
Für jede eingetragene Note wird geprüft,
ob es die jeweils beste der/des aktuellen Studierenden ist.
\end{note}

\begin{enumerate}[a)]

  \item \label{a} Bewerten Sie die Anfrage im Hinblick auf das zu erwartende Leistungsverhalten
    und die Optimierungsmöglichkeiten.

\begin{solution}
Aus Sicht des Leistungsverhaltens ist diese Anfrage problematisch,
weil die Unteranfrage für jedes einzelne Studierende-Noten-Paar
komplett wiederholt werden muss.
\end{solution}

  \item Schreiben Sie die Anfrage so um, dass es die unter \ref{a}) genannten
    Probleme löst und die Bearbeitung optimiert werden kann.

\cprotEnv
\begin{solution}
Das Problem ist, wie bereits oben genannt, die Abhängigkeit der Unteranfrage
von \texttt{s.matrikelnr} der oberen Anfrage.
Es handelt sich also, wie im ersten Übungsblatt bereits kennengelernt,
um eine \emph{korrelierte Unteranfrage}.

Ziel ist es also, die Anfrage so umzuschreiben,
dass die Abhängigkeit eliminiert wird.
Das ist in diesem Fall relativ einfach.
Die Grundidee besteht darin, die beste Note jedes Studierenden bereits
in der Notentabelle zu ermitteln und das Ergebnis mithilfe eines Kreuzprodukts
mit der Studierenden-Relation zu verbinden:

\begin{lstlisting}
SELECT s.name, n.pruefungsnr, n.note
FROM   Studierende s, Noten n,
       (SELECT  n2.matrikelnr as matrikelnr,
                min(n2.note) as beste
        FROM    Noten n2
        GROUP BY n2.matrikelnr) m
WHERE  s.matrikelnr = n.matrikelnr AND
       s.matrikelnr = m.matrikelnr AND
       n.note = m.beste;
\end{lstlisting}

\cprotEnv
\begin{note}
Die ganze Anfrage lässt sich auch alternativ komplett ohne Unteranfrage schreiben:

\begin{lstlisting}
SELECT   s.name, n.pruefungsnr, n.note
FROM     Studierende s, Noten n, Noten n2
WHERE    n.matrikelnr = n2.matrikelnr
AND      s.matrikelnr = n.matrikelnr
GROUP BY n.matrikelnr, s.name, n.pruefungsnr, n.note
HAVING   n.note = min(n2.note)
\end{lstlisting}
\end{note}

Nun muss die Unteranfrage nur noch einmal evaluiert werden.
Das Ergebnis kann mittels bereits kennengelernter JOIN-Mechanismen dann
mit den beiden Relationen \texttt{Studierende} und \texttt{Noten} verknüpft werden.

Diese Optimierung kann so vom DBMS automatisch durchgeführt werden.
Sie ist vergleichsweise leicht erkennbar und wie man sieht, wurde die Unteranfrage
nur in die FROM-Klausel verschoben und durch ein Symbol (\texttt{m.beste}) ersetzt.

Wird eine korrelierte Unteranfrage auf diese Art aufgelöst,
spricht man von \emph{Unnesting}.
\end{solution}

\end{enumerate}
