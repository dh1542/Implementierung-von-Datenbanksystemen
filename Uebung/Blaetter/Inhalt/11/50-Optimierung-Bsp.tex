\section{Optimierung im Betrieb -- Beispiele}

Große Datenbanksysteme wie beispielsweise Oracle bieten Administratoren die Möglichkeit,
im Betrieb Einfluss auf bestimmte Optimierungsparameter zu nehmen.
Welche Informationen benötigt der Administrator und wie könnten Schnittstellen für diese Einflussnahme aussehen?

\begin{note}
Diskussionsaufgabe! Im folgenden ist beispielhaft beschrieben,
welche Möglichkeiten Oracle unter anderem bietet.

Darüber hinaus sind viele weitere denkbar und auch implementiert.
\end{note}

\cprotEnv
\begin{solution}
Wie Sie in den letzten Blättern kennengelernt haben,
nutzen Datenbanksysteme häufig bereits verschiedenste Algorithmen,
um das Leistungsverhalten zu steigern.
Dazu müssen verschiedene Hilfsstrukturen mitgeführt werden.

Der Administrator als Mensch hat unter Umständen durch Kenntnis
des gesamten Systems (also Hardware, Betriebssystem, Anwendung etc.)
die Möglichkeit, bessere Aussagen zu möglichen Optimierungen zu treffen,
als es das Datenbankverwaltungssystem durch reines Mitführen von Statistiken kann.

Damit der Administrator die vom System getroffenen Entscheidungen
jedoch bewerten kann, benötigt er Zugriff auf die internen Statistiken,
auf Basis derer das Datenbankverwaltungssystem die Entscheidungen trifft.

    Bei Oracle ist es nicht nur möglich,
    die internen Statistiken über die Datenbank selbst zu betrachten.
    Es werden darüber hinaus auch Statistiken
    über die ausgeführten SQL-Anfragen gesammelt,
    wie z.\,B. Ausführungszeiten, generierte Ausführungspläne
    etc.\footnote{\url{http://docs.oracle.com/cd/B19306_01/server.102/b14211/sql_1016.htm\#i26072}}

    Stellt der Administrator nun fest, dass für eine bestimmte Anfrage
    regelmäßig ein (in der Praxis) schlechter Ausführungsplan gewählt wird,
    kann er z.B. mit sogenannten \emph{HINTS} die Ausführung beeinflussen.
    Dabei wird in die SQL-Anfrage selbst ein Kommentar geschrieben,
    der vom Optimierer interpretiert
    wird.\footnote{\url{http://docs.oracle.com/cd/B19306_01/server.102/b14211/hintsref.htm}}

    Ein solcher HINT könnte z.\,B. so aussehen:

\begin{lstlisting}
    SELECT /*+ USE_NL(e d) */ e.fname, e.lname, e.addr
        FROM employee e
        JOIN department d
        ON d.num = e.dep
        WHERE d.name = 'Research';
\end{lstlisting}

    Damit wird dem Optimierer empfohlen, einen Nested-Loop-Join zu verwenden.
    Zu beachten ist jedoch, dass es sich dabei
    -- wie der Begriff \emph{HINT} schon andeutet --
    nicht um eine Anweisung handelt,
    sondern wie gesagt vielmehr um eine \textbf{Empfehlung}.
    Der Optimierer kann sich trotzdem darüber hinwegsetzen.
    Der Administrator erkennt dies dann wieder in den Statistiken.


    Das ständige (oder regelmäßige) Mitführen von Statistiken generiert
    auf dem System allerdings zusätzliche Last,
    entweder durch ständiges Aktualisieren von Statistikdaten
    oder durch regelmäßige Aktualisierungsläufe.

    Oracle bietet daher einen weiteren ganz grundsätzlichen Ansatz an.
    Es ist möglich, den kostenbasierten Optimierer zu deaktivieren
    und Ausführungspläne z.\,B. nach festen Regeln erstellen zu
    lassen.\footnote{\url{http://docs.oracle.com/cd/B19306_01/server.102/b14211/optimops.htm}}
    Das bietet den Vorteil, dass auf das Mitführen von Statistiken
    (vor allem wie viele Tupel in einer Relation sind, Selektivitäten etc.)
    weitgehend verzichtet werden kann.
    Ist also absehbar,
    dass der mit den Statistiken verbundene Aufwand größer ist
    als der Nutzen der kostenbasierten Wahl der Ausführungspläne,
    kann auch
    auf den kostenbasierten Optimierer komplett verzichtet werden.

    Weitere mögliche Ideen, unabhängig vom verwendeten System:
    \begin{itemize}
        \item Wenn häufig eine komplizierte JOIN-Abfrage durchgeführt wird:
            \textbf{Materialisierte Sicht} anlegen
            (und die Anfragen daran anpassen oder dem Optimierer beibringen,
            dass er die Sicht automatisch nutzen kann)
        \item Wenn in der Selektion häufig ein Ausdruck verwendet wird,
            der die Nutzung eines bisher verfügbaren Index unmöglich macht:
            \textbf{Function-Based Index} über diesen Ausdruck anlegen
        \item Puffer vergrößern
        \item schnellere / besser geeignete Hintergrundspeicher
    \end{itemize}
\end{solution}
