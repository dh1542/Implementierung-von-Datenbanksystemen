\section{Programmieraufgabe 2: SeqRecordFile}

\subsection{Aufgabenstellung}
\begin{enumerate}
	\item Implementieren Sie eine Klasse, die die Schnittstelle \beamertxt{\linebreak}\texttt{idb.record.SeqRecordFile} implementiert.
		Beachten Sie die Dokumentation der Methoden in der Schnittstelle.
	\item Tragen Sie den Konstruktor Ihrer Klasse in \texttt{idb.construct.Util} in der Methode \texttt{generateSeqRecordFile()} ein.
		Die Funktion soll den DBBuffer und eine BlockFile nehmen und eine \textbf{neue} SeqRecordFile zurückgeben.
		Beachten Sie, dass die BlockFile zu diesem Zeitpunkt schon geöffnet ist.
	\item Tragen Sie den Konstruktor Ihrer Klasse in \texttt{idb.construct.Util} in der Methode \texttt{rebuildSeqRecordFile()} ein.
		Die Funktion soll den DBBuffer und eine BlockFile nehmen und eine \textbf{existierende} SeqRecordFile vom Laufwerk laden.
		Beachten Sie, dass die BlockFile zu diesem Zeitpunkt schon geöffnet ist.
	\item Sorgen Sie dafür, dass Sie alle Tests aus der Klasse \texttt{SeqTests} erfüllen.
	Sie können diese Testfälle mit \lstinline|ant Meilenstein2| ausführen.
	\item Die Abgabe auf GitLab erfolgt zeitgleich mit der Abgabe der Zusatzaufgaben des nächsten Übungsblattes auf StudOn. Markieren Sie hierfür ihre Abgabe mit dem Tag "`Aufgabe-2"'.
\end{enumerate}

\subsection{Hinweise}
\begin{itemize}
	\item Im Gegensatz zur Vorlesung ist unsere SeqRecordfile eine List von Tupeln, keine Menge. Das bedeutet, dass die Ordnung erhalten bleiben muss.
	\item Achten Sie darauf, dass Sätze größer als ein Block sein können.
	\item Achten Sie darauf, dass Sie für das Auslesen der fragmentierten Sätze die Länge bereits kennen müssen.
		Es empfiehlt sich, die Länge der Sätze vor den eigentlichen Daten abzuspeichern.
	\item Zum Zugriff auf die BlockFile soll der \texttt{DBBuffer} verwendet werden.
	\item Ein \texttt{DataObject} ist eine Klasse, die sich um die Serialisierung und Deserialisierung kümmert.
	\item Ein \texttt{DataObject::size} kann sich beim Aufruf von \texttt{DataObject::read} verändern.
	\item Beim Aufruf von \texttt{DBString::copy} muss nicht zwangsläufig ein DBString zurückkommen, sondern ein beliebiges \texttt{DataObject}.
			Achten Sie darauf, dass ein Cast nicht erfolgreich sein muss.
	\item Beim fragmentierten Lesen eines \texttt{DataObjects} muss eine Liste von Tripel übergeben werden.
		Achten Sie darauf, keine \textit{use-after-free} nach \texttt{DBBuffer::unfix} zu begehen. (Dies wird die Testfälle fehlschlagen lassen.)
	\item Testen Sie Ihre Implementierung, indem Sie den Inhalt der Klasse \texttt{idb.Main} durch Testcode für ihre SeqRecordFile ersetzen.
		Setzen Sie \texttt{Main} auf den orginalen Zustand zurück, sobald Sie sich von der Korrektheit überzeugt haben.
	\item Zur Implementierung der View bietet sich die Verwendung einer inneren Klasse an, um Zugriff auf den Zustand der SeqRecordFile zu haben.
	\item Um das Schreiben von Tupeln zu beschleunigen, lohnt es sich, im ersten Block Metadaten abzulegen, in welchem Block an welcher Stelle der freie Platz anfängt.
	\item SeqRecordFile bietet keine Operation zum Löschen an und ist deswegen für die langfristige Speicherung nicht nützlich, sondern nur für spezifische Aufgaben in der Datenbank gedacht.
\end{itemize}
