\section{Fragen zur Vorlesung}
\begin{enumerate}[a)]
	\item Wozu dient die Abstraktion von Blöcken zu Sätzen?

	\begin{solution}
	Blöcke sind eine technische Abstraktion, die von der Hardware vorgegeben ist. Die Anwendung interessiert sich aber nicht für technische Details, sondern für Anwendungsgegenstände. So einen Anwendungsgegenstand nennen wir Satz. Insbesondere können Sätze variable Länge haben. Außerdem kann man z.\,B. die Blockgrößen ändern oder auf ein Gerät mit anderer Blockgröße umstellen, ohne dass die Anwendung von dieser Änderung betroffen ist.
	\end{solution}


	\item Wie funktioniert die sequenzielle Satzdatei und welche Einschränkungen ergeben sich hieraus?

	\begin{solution}
	Eine Satzdatei ist eine Menge von Sätzen mit fester oder variabler Länge ohne bestimmte Reihenfolge (d.\,h. ohne Sortierung).

	Die sequenzielle Satzdatei zeichnet sich dadurch aus, dass kein wahlfreier Zugriff möglich ist, Löschen und Einfügen nur am Ende.
	Man kann also nur von vorne nach hinten lesen oder schreiben, sowie Anfügen am Ende. Für Caches ist das ein Vorteil: Man weiß genau, welcher Block der nächste ist. Die Schreibreihenfolge entspricht also der Abspeicherungsreihenfolge und der Lesereihenfolge.
	\end{solution}


	\item Wann bezeichnet man eine Satzadresse als stabil?

	\begin{solution}
	Wenn sie beim Verschieben des Satzes gleich bleibt.

	Das ist wichtig, wenn man Sätze umorganisieren können möchte (notwendig z.\,B. bei Größenänderungen) und man die Satzadressen nach außen weitergibt. Denn ein Aufrufer erwartet (zu Recht), dass er einen Satz unter einer einmal vergebenen Adresse auch wiederfindet. In die höheren Schichten muss man eine Satzadresse immer weitergeben (z.\,B. für die Zugriffspfade). Allerdings wird das später wieder vor den noch höheren Schichten verborgen. Der Anwender bekommt normalerweise keine Satzadressen zu sehen. Eine Ausnahme ist z.\,B. die Oracle-ROWID.
	\end{solution}


	\item Welche Aussagen über Satzadressen sind \textit{Wahr}, welche \textit{Falsch}?

    \begin{description}
        \itemmc    Satzadressen sind zur Realisierung von sequenziellen Satzdateien nötig.
        \itemmcsol Satzadressen ermöglichen wahlfreien Zugriff auf Sätze.
        \itemmc    Eine Satzadresse ist ein vom Anwender gewählter Identifikator.
        \itemmcsol Eine Satzadresse kann wiederverwendet werden.
        \itemmcsol Die Satzadresse kann sich bei Verschiebung des zugehörigen Satzes ändern.
    \end{description}

	\begin{solution}
	Eine Satzadresse allgemein ist ein Bezeichner, unter der man einen einzelnen Satz wiederfinden kann. Random Access ist nur so möglich. Eine Satzadresse wird immer vom System vergeben. Wenn man einen Satz einfügt, bekommt man dafür eine Adresse. Gibt man dem System diese Adresse wieder, findet es den Satz wieder.
	
	\begin{itemize}
		\item Bei der sequentiellen Satzdatei gibt es somit keine Satzadressen.
		\item Ein wahlfreier Zugriff (Direktzugriff) ist somit über die Satzadresse möglich.
		\item Eine Satzadresse wird somit vom System bestimmt.
		\item Einmal vergebene Satzadressen sollten auch nach dem Löschen des Satzes nicht wiederverwendet werden, da irgendwo noch eine Referenz darauf existieren könnte. So kann nicht unterschieden werden, ob dort noch der alte Satz liegt, oder ob der Satz gelöscht wurde, und jetzt einfach ein neuer Satz dort liegt.
			Das System muss also anders sicherstellen, dass diese beiden Fälle unterscheidbar bleiben.
		\item Beim Verschieben kann sich die Satzadresse ändern, da sich die Aktualisierung der Satzadresse in höheren Schichten jedoch als schwer erweist, sollte sie sich nicht ändern. Siehe hierzu \textit{stabil}.
	\end{itemize}
	\end{solution}
\end{enumerate}



