\section{Einfügen und Löschen im B-Baum}

\begin{enumerate}[a)]
	\item Fügen Sie in einen anfangs leeren B-Baum mit $k=2$ die Zahlen eins bis zwanzig in aufsteigender Reihenfolge ein. Was fällt Ihnen dabei auf?

\begin{solution}
Die Zahlen 1 bis 4 lassen sich problemlos in den Wurzelknoten einfügen.
\begin{center}
    \begin{tikzpicture}[
            start chain=0 going right,
            defaultNode/.style={defaultNode1},
        ]

        %Level0
		\draw pic {firstInnerNode={1}{2}{3}{4}{0}{0}{2}};
    \end{tikzpicture}
\end{center}

Bei 5 kommt es zum Wurzelsplitt und der Baum wächst um eins. Danach steht 3 in der Wurzel, 1 und 2 im linken Blattknoten und 4 und 5 im rechten Blattknoten.

\begin{center}
    \begin{tikzpicture}[
            start chain=0 going right,
            defaultNode/.style={defaultNode1},
        ]

        %Level0
		\draw pic {firstInnerNode={3}{}{}{}{0}{0}{5}};

        %Level1
        \draw pic {firstInnerNode={1}{2}{}{}{1}{1}{2}};
        \draw pic {innerNode={4}{5}{}{}{1}{2}};

        %Verbindungspfeile 0 - 1
        \draw pic {connect={0}{0}{1}};
        \draw pic {connect={0}{1}{2}};
    \end{tikzpicture}
\end{center}

6 und 7 können wieder ohne Splitt in den rechten Blattknoten eingefügt werden.

\begin{center}
    \begin{tikzpicture}[
            start chain=0 going right,
            defaultNode/.style={defaultNode1},
        ]

        %Level0
        \draw pic {firstInnerNode={3}{}{}{}{0}{0}{5}};

        %Level1
        \draw pic {firstInnerNode={1}{2}{}{}{1}{1}{2}};
        \draw pic {innerNode={4}{5}{6}{7}{1}{2}};

        %Verbindungspfeile 0 - 1
        \draw pic {connect={0}{0}{1}};
        \draw pic {connect={0}{1}{2}};
    \end{tikzpicture}
\end{center}

Das Einfügen der 8 führt zum Überlauf im rechten Blattknoten.
Der Knoten wird also an der 6 gesplittet, die in den Wurzelknoten wandert:

\begin{center}
    \begin{tikzpicture}[
            start chain=0 going right,
            defaultNode/.style={defaultNode1},
        ]

        %Level0
        \draw pic {firstInnerNode={3}{6}{}{}{0}{0}{8}};

        %Level1
        \draw pic {firstInnerNode={1}{2}{}{}{1}{1}{2}};
        \draw pic {innerNode={4}{5}{}{}{1}{2}};
        \draw pic {innerNode={7}{8}{}{}{1}{3}};

        %Verbindungspfeile 0 - 1
        \draw pic {connect={0}{0}{1}};
        \draw pic {connect={0}{1}{2}};
        \draw pic {connect={0}{2}{3}};
    \end{tikzpicture}
\end{center}

9 und 10 können wieder ohne Splitt eingefügt werden.
Beim Einfügen der 11 wird der rechte Blattknoten wieder gesplittet,
9 wird in den Wurzelknoten gezogen.
Das gleiche gilt für das Einfügen von 12, 13 und 14, nun ist der Wurzelknoten voll:

\begin{center}
    \begin{tikzpicture}[
            start chain=0 going right,
            defaultNode/.style={defaultNode2},
        ]

        %Level0
        \draw pic {firstInnerNode={3}{6}{9}{12}{0}{0}{8}};

        %Level1
        \draw pic {firstInnerNode={1}{2}{}{}{1}{1}{2}};
        \draw pic {innerNodeNarrow={4}{5}{}{}{1}{2}};
        \draw pic {innerNodeNarrow={7}{8}{}{}{1}{3}};
        \draw pic {innerNodeNarrow={10}{11}{}{}{1}{4}};
        \draw pic {innerNodeNarrow={13}{14}{}{}{1}{5}};

        %Verbindungspfeile 0 - 1
        \draw pic {connect={0}{0}{1}};
        \draw pic {connect={0}{1}{2}};
        \draw pic {connect={0}{2}{3}};
        \draw pic {connect={0}{3}{4}};
        \draw pic {connect={0}{4}{5}};
    \end{tikzpicture}
\end{center}

15 und 16 können wieder einfach in den rechten Blattknoten eingefügt werden.
Das Einfügen der 17 führt zum Splitt im rechten Blattknoten.
Da der Wurzelknoten aber bereits voll ist, führt das auch zum Wurzelsplitt.
Der entstehende Baum sieht dann so aus:

\begin{center}
    \begin{tikzpicture}[
            start chain=0 going right,
            defaultNode/.style={defaultNode2},
        ]

        %Level0
        \draw pic {firstInnerNode={9}{}{}{}{0}{0}{5}};

        %Level1
        \draw pic {firstInnerNode={3}{6}{}{}{1}{1}{2}};
        \draw pic {innerNodeVar={12}{15}{}{}{1}{2}{4}};

        %Level2
        \draw pic {firstInnerNode={1}{2}{}{}{2}{3}{0}};
        \draw pic {innerNodeNarrow={7}{8}{}{}{2}{5}};
        \draw pic {innerNode={10}{11}{}{}{2}{6}};
        \draw pic {innerNodeNarrow={16}{17}{}{}{2}{8}};

        %Level2.5
        \draw pic {firstInnerNode={4}{5}{}{}{3}{4}{3}};
        \draw pic {innerNodeVar={13}{14}{}{}{2}{7}{5.2}};


        %Verbindungspfeile
        \draw pic {connect={0}{0}{1}};
        \draw pic {connect={0}{1}{2}};
        \draw pic {connect={1}{0}{3}};
        \draw pic {connect={1}{1}{4}};
        \draw pic {connect={1}{2}{5}};
        \draw pic {connect={2}{0}{6}};
        \draw pic {connect={2}{1}{7}};
        \draw pic {connect={2}{2}{8}};
    \end{tikzpicture}
\end{center}

18 und 19 lassen sich nun wieder einfach in den letzten Blattknoten einfügen,
bei der 20 kommt es erneut zu einem Überlauf und einem Splitt.

Am Ende erhält man folgenden B-Baum:

\begin{center}
    \begin{tikzpicture}[
            start chain=0 going right,
            defaultNode/.style={defaultNode2},
        ]

        %Level0
        \draw pic {firstInnerNode={9}{}{}{}{0}{0}{6}};

        %Level1
        \draw pic {firstInnerNode={3}{6}{}{}{1}{1}{2.5}};
        \draw pic {innerNodeVar={12}{15}{18}{}{1}{2}{4.5}};

        %Level2
        \draw pic {firstInnerNode={1}{2}{}{}{2}{3}{0.5}};
        \draw pic {innerNodeVar={7}{8}{}{}{2}{5}{0.4}};
        \draw pic {innerNodeVar={13}{14}{}{}{2}{7}{2.4}};
        \draw pic {innerNodeVar={19}{20}{}{}{2}{9}{0.8}};

        %Level2.5
        \draw pic {firstInnerNode={4}{5}{}{}{3}{4}{3.5}};
        \draw pic {innerNodeVar={10}{11}{}{}{3}{6}{0.6}};
        \draw pic {innerNodeVar={16}{17}{}{}{3}{8}{2.7}};


        %Verbindungspfeile
        \draw pic {connect={0}{0}{1}};
        \draw pic {connect={0}{1}{2}};
        \draw pic {connect={1}{0}{3}};
        \draw pic {connect={1}{1}{4}};
        \draw pic {connect={1}{2}{5}};
        \draw pic {connect={2}{0}{6}};
        \draw pic {connect={2}{1}{7}};
        \draw pic {connect={2}{2}{8}};
        \draw pic {connect={2}{3}{9}};
    \end{tikzpicture}
\end{center}

Es fällt auf, dass der B-Baum nahezu minimale Auslastung aufweist. Dies liegt am sequentiellen Einfügen einer aufsteigenden Zahlenfolge in den B-Baum: Nach dem Aufspalten eines Knotens werden in den Knoten, der die kleineren Datensätze enthält, keine weiteren Werte mehr eingefügt.

Es fällt außerdem auf, dass in der Wurzelebene und in den inneren Ebenen jeweils immer nur Vielfache einer bestimmten Zahl $x$ auftreten. Bei genauerem Betrachten findet man heraus, dass sich diese Zahl $x$ für jede Ebene $i$ folgendermaßen berechnen lässt:
\[x = (k + 1)^{h - i}, i \in \mathbb{N}_{>0},\; h: \mathrm{H"ohe\ des\ Baums}\]

Erläuterung:
\begin{enumerate}[\arabic{enumii}.]
\item Wir stellen fest, dass alle „linken Kindblöcke“ durch das Splitten in unserem Beispiel immer zwei (also $k$) Elemente pro Block beinhalten.
\item Der linke Unterbaum besitzt in der ersten Ebene als Wurzel einen Knoten, welcher in der Ebene unter unserer betrachteten Ebene liegt.
Somit beinhaltet dieser zwei Elemente und drei Nachfolgeknoten, es sei denn, es handelt sich um einen Blattknoten.
\item Jeder dieser drei Nachfolger in der zweiten Ebene beinhaltet wieder zwei Elemente und hat damit drei Nachfolger, wenn es sich nicht um einen Blattknoten handelt.
Damit befinden sich in der dritten Ebene $3^2 = 9$ Knoten.
\item Führt man nun die Reihe fort kommt man mit der Unterbaumhöhe u auf die folgende Formel für die Anzahl der Elemente des linken Unterbaums:
\[(1 + 3 +3^2 +3^3 \ldots + 3^{u-1}) \cdot 2\]
\item Durch $u:= h-i$ mit h als Gesamthöhe und i als Ebene des betrachteten Knotens ergibt sich folgende allgemeine Summe:
\[(1 + (k+1) + (k+1)^2 + (k+1)^3 + \ldots + (k+1)^{h-i-1}) \cdot k = \left(\sum_{j=0}^{h-i-1} (k+1)^j\right)\cdot k \]
\item Das "`linkeste Element"' in unserer betrachteten Ebene ist damit dann das nächstgrößere Element, was sich mithilfe der geometrischen Reihe ($\sum_{i=0}^{n-1}q^i = \frac{q^n - 1}{q - 1}$) folgendermaßen berechnen lässt:
\begin{align*}
Wert &:= \text{Elemente im Unterbaum} + 1\\
& = \left(\sum_{j=0}^{h-i-1} (k+1)^j\right)\cdot k +1 \stackrel{\text{Geometrische Reihe}}{=} \frac{(k+1)^{h-i}-1}{k + 1 - 1}\cdot k + 1\\
&= (k+1)^{h-i}- 1 +1 = (k+1)^{h-i}
\end{align*}
\item Zwischen diesem "`linkesten Element"' und dem nächsten Element der Ebene befindet sich nun wieder ein Unterbaum, welcher die selbe Höhe wie der erste Unterbaum besitzt.
Damit enthält dieser wieder $(k+1)^{h-i} - 1$ Elemente und somit lautet der Wert des Elements $2\cdot (k+1)^{h-i}$.
\item Somit befinden sich in einer Ebene nur Vielfache des "`linkesten Elements"'.
\end{enumerate}
\end{solution}

\item Löschen Sie nun den Eintrag 9 und zeichnen Sie den entstehenden Baum.

\begin{solution}
Da ein innerer Eintrag gelöscht wird, muss er durch den direkten Vorgänger oder Nachfolger ersetzt werden.
Da Vorgänger- und Nachfolgerblattknoten gleich viele Einträge haben, können wir uns frei zwischen Vorgänger und Nachfolger entscheiden. Wir nehmen hier den Vorgänger 8:

\begin{note}
	Wir nehmen die 8, weil die 10 einfacher ist, da im rechten Teilbaum die 18 exisitiert, und so kein Wurzel-merge n\"otig ist.
\end{note}

\begin{center}
    \begin{tikzpicture}[
            start chain=0 going right,
            defaultNode/.style={defaultNode2},
        ]

        %Level0
        \draw pic {firstInnerNode={8}{}{}{}{0}{0}{6}};

        %Level1
        \draw pic {firstInnerNode={3}{6}{}{}{1}{1}{2.5}};
        \draw pic {innerNodeVar={12}{15}{18}{}{1}{2}{4.5}};

        %Level2
        \draw pic {firstInnerNode={1}{2}{}{}{2}{3}{0.5}};
        \draw pic {innerNodeVar={7}{}{}{}{2}{5}{0.4}};
        \draw pic {innerNodeVar={13}{14}{}{}{2}{7}{2.6}};
        \draw pic {innerNodeVar={19}{20}{}{}{2}{9}{0.6}};

        %Level2.5
        \draw pic {firstInnerNode={4}{5}{}{}{3}{4}{3.5}};
        \draw pic {innerNodeVar={10}{11}{}{}{3}{6}{0.6}};
        \draw pic {innerNodeVar={16}{17}{}{}{3}{8}{2.7}};


        %Verbindungspfeile
        \draw pic {connect={0}{0}{1}};
        \draw pic {connect={0}{1}{2}};
        \draw pic {connect={1}{0}{3}};
        \draw pic {connect={1}{1}{4}};
        \draw pic {connect={1}{2}{5}};
        \draw pic {connect={2}{0}{6}};
        \draw pic {connect={2}{1}{7}};
        \draw pic {connect={2}{2}{8}};
        \draw pic {connect={2}{3}{9}};
    \end{tikzpicture}
\end{center}


Dadurch entsteht ein Unterlauf im Blattknoten,
der deshalb mit dem Elternknoten gemischt werden muss:

\begin{center}
    \begin{tikzpicture}[
            start chain=0 going right,
            defaultNode/.style={defaultNode2},
        ]

        %Level0
        \draw pic {firstInnerNode={8}{}{}{}{0}{0}{6}};

        %Level1
        \draw pic {firstInnerNode={3}{}{}{}{1}{1}{2.5}};
        \draw pic {innerNodeVar={12}{15}{18}{}{1}{2}{4.5}};

        %Level2
        \draw pic {firstInnerNode={1}{2}{}{}{2}{3}{0.5}};
        \draw pic {innerNodeVar={4}{5}{6}{7}{3}{4}{0.5}};
        \draw pic {innerNodeVar={13}{14}{}{}{2}{7}{1.7}};
        \draw pic {innerNodeVar={19}{20}{}{}{2}{9}{0.8}};

        %Level2.5
        \draw pic {firstInnerNode={10}{11}{}{}{3}{6}{6.6}};
        \draw pic {innerNodeVar={16}{17}{}{}{3}{8}{2.9}};


        %Verbindungspfeile
        \draw pic {connect={0}{0}{1}};
        \draw pic {connect={0}{1}{2}};
        \draw pic {connect={1}{0}{3}};
        \draw pic {connect={1}{1}{4}};
        \draw pic {connect={2}{0}{6}};
        \draw pic {connect={2}{1}{7}};
        \draw pic {connect={2}{2}{8}};
        \draw pic {connect={2}{3}{9}};
    \end{tikzpicture}
\end{center}

Nun besteht aber im linken inneren Knoten ein Unterlauf, der aufgelöst werden muss.
Dazu wird mit der Wurzel gemischt. Es entsteht ein Überlauf in der (neuen) Wurzel,
die deshalb gesplittet werden muss. Ergebnis:

\begin{center}
    \begin{tikzpicture}[
            start chain=0 going right,
            defaultNode/.style={defaultNode2},
        ]

        %Level0
        \draw pic {firstInnerNode={12}{}{}{}{0}{0}{6}};

        %Level1
        \draw pic {firstInnerNode={3}{8}{}{}{1}{1}{2.5}};
        \draw pic {innerNodeVar={15}{18}{}{}{1}{2}{4.5}};

        %Level2
        \draw pic {firstInnerNode={1}{2}{}{}{2}{3}{0}};
        \draw pic {innerNodeVar={10}{11}{}{}{2}{5}{0.4}};
        \draw pic {innerNodeVar={13}{14}{}{}{2}{7}{1}};
        \draw pic {innerNodeVar={19}{20}{}{}{2}{9}{0.8}};

        %Level2.5
        \draw pic {firstInnerNode={4}{5}{6}{7}{3}{4}{2.6}};
        \draw pic {innerNodeVar={16}{17}{}{}{3}{8}{4.9}};


        %Verbindungspfeile
        \draw pic {connect={0}{0}{1}};
        \draw pic {connect={0}{1}{2}};
        \draw pic {connect={1}{0}{3}};
        \draw pic {connect={1}{1}{4}};
        \draw pic {connect={1}{2}{5}};
        \draw pic {connect={2}{0}{7}};
        \draw pic {connect={2}{1}{8}};
        \draw pic {connect={2}{2}{9}};
    \end{tikzpicture}
\end{center}

\end{solution}

\end{enumerate}
