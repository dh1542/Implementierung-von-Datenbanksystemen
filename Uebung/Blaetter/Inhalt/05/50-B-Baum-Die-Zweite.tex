\section{B-Baum die Zweite}
Fügen Sie nun die folgenden 20 Zahlen in der vorgegebenen Reihenfolge in einen leeren B-Baum mit \( k=2 \) ein:\\
3, 14, 15, 92, 65, 35, 89, 79, 32, 38, 46, 26, 43, 50, 28, 84, 19, 71, 69, 39

\begin{note}
	Die Zahlen 3, 14, 15, 92 lassen sich problemlos in den Wurzelknoten einfügen.
	\begin{center}
		\begin{tikzpicture}[
		start chain=0 going right,
		defaultNode/.style={defaultNode1},
		]

		%Level0
		\draw pic {firstInnerNode={3}{14}{15}{92}{0}{0}{2}};
		\end{tikzpicture}
	\end{center}

Bei 65 kommt es zum Wurzelsplitt und der Baum wächst um eins.
Danach stehen 15 in der Wurzel, 3 und 14 im linken Blattknoten und 65 und 92 im rechten Blattknoten.

	\begin{center}
		\begin{tikzpicture}[
		start chain=0 going right,
		defaultNode/.style={defaultNode1},
		]

		%Level0
		\draw pic {firstInnerNode={15}{}{}{}{0}{0}{5}};

		%Level1
		\draw pic {firstInnerNode={3}{14}{}{}{1}{1}{2}};
		\draw pic {innerNode={65}{92}{}{}{1}{2}};

		%Verbindungspfeile 0 - 1
		\draw pic {connect={0}{0}{1}};
		\draw pic {connect={0}{1}{2}};
		\end{tikzpicture}
	\end{center}

35 und 89 können wieder ohne Splitt in den rechten Blattknoten eingefügt werden, bis es dann bei der 79 zu einem Splitt kommt.

	\begin{center}
		\begin{tikzpicture}[
		start chain=0 going right,
		defaultNode/.style={defaultNode1},
		]

		%Level0
		\draw pic {firstInnerNode={15}{79}{}{}{0}{0}{5}};

		%Level1
		\draw pic {firstInnerNode={3}{14}{}{}{1}{1}{2}};
		\draw pic {innerNode={35}{65}{}{}{1}{2}};
		\draw pic {innerNode={89}{92}{}{}{1}{3}};

		%Verbindungspfeile 0 - 11
		\draw pic {connect={0}{0}{1}};
		\draw pic {connect={0}{1}{2}};
		\draw pic {connect={0}{2}{3}};
		\end{tikzpicture}
	\end{center}

32 und 38 können wieder ohne Probleme in den mittleren Blattknoten eingefügt werden. Erst bei der 46 kommt es wieder zu einem Überlauf.

	\begin{center}
		\begin{tikzpicture}[
		start chain=0 going right,
		defaultNode/.style={defaultNode1},
		]

		%Level0
		\draw pic {firstInnerNode={15}{38}{79}{}{0}{0}{8}};

		%Level1
		\draw pic {firstInnerNode={3}{14}{}{}{1}{1}{2}};
		\draw pic {innerNode={32}{35}{}{}{1}{2}};
		\draw pic {innerNode={46}{65}{}{}{1}{3}};
		\draw pic {innerNode={89}{92}{}{}{1}{4}};

		%Verbindungspfeile 0 - 1
		\draw pic {connect={0}{0}{1}};
		\draw pic {connect={0}{1}{2}};
		\draw pic {connect={0}{2}{3}};
		\draw pic {connect={0}{3}{4}};
		\end{tikzpicture}
	\end{center}

	26, 43, 50, 28 und 84 können wieder ohne Splitt eingefügt werden.
	Beim Einfügen der 19 wird der zweite Blattknoten gesplittet,
	28 wird in den Wurzelknoten gezogen.

	\begin{center}
		\begin{tikzpicture}[
		start chain=0 going right,
		defaultNode/.style={defaultNode2},
		]

		%Level0
		\draw pic {firstInnerNode={15}{28}{38}{79}{0}{0}{8}};

		%Level1
		\draw pic {firstInnerNode={3}{14}{}{}{1}{1}{2}};
		\draw pic {innerNodeNarrow={19}{26}{}{}{1}{2}};
		\draw pic {innerNodeNarrow={32}{35}{}{}{1}{3}};
		\draw pic {innerNodeNarrow={43}{46}{50}{65}{1}{4}};
		\draw pic {innerNodeNarrow={84}{89}{92}{}{1}{5}};

		%Verbindungspfeile 0 - 1
		\draw pic {connect={0}{0}{1}};
		\draw pic {connect={0}{1}{2}};
		\draw pic {connect={0}{2}{3}};
		\draw pic {connect={0}{3}{4}};
		\draw pic {connect={0}{4}{5}};
		\end{tikzpicture}
	\end{center}

Das Einfügen der 71 führt zum Splitt im Blattknoten. die 50 wird nach oben geschoben und es kommt zum Überlauf im Wurzelknoten. Dieser wird gesplittet und die 38 wandert nach oben.
Der entstehende Baum sieht dann so aus:

	\begin{center}
		\begin{tikzpicture}[
		start chain=0 going right,
		defaultNode/.style={defaultNode2},
		]

		%Level0
		\draw pic {firstInnerNode={38}{}{}{}{0}{0}{8}};

		%Level1
		\draw pic {firstInnerNode={15}{28}{}{}{1}{1}{4}};
		\draw pic {innerNodeVar={50}{79}{}{}{1}{2}{4}};

		%Level2
		\draw pic {firstInnerNode={3}{14}{}{}{2}{3}{1}};
		\draw pic {innerNodeVar={32}{35}{}{}{2}{5}{0.8}};
		\draw pic {innerNodeNarrow={43}{46}{}{}{2}{6}};
		\draw pic {innerNodeVar={84}{89}{92}{}{2}{8}{0.8}};

		%Level2.5
		\draw pic {firstInnerNode={19}{26}{}{}{2.5}{4}{4}};
		\draw pic {innerNodeVar={65}{71}{}{}{2.5}{7}{4}};

		%Verbindungspfeile
		\draw pic {connect={0}{0}{1}};
		\draw pic {connect={0}{1}{2}};
		\draw pic {connect={1}{0}{3}};
		\draw pic {connect={1}{1}{4}};
		\draw pic {connect={1}{2}{5}};
		\draw pic {connect={2}{0}{6}};
		\draw pic {connect={2}{1}{7}};
		\draw pic {connect={2}{2}{8}};
		\end{tikzpicture}
	\end{center}

Final werden nun die 69 und 39 eingefügt.

Am Ende erhält man folgenden B-Baum:

	\begin{center}
	\begin{tikzpicture}[
	start chain=0 going right,
	defaultNode/.style={defaultNode2},
	]

	%Level0
	\draw pic {firstInnerNode={38}{}{}{}{0}{0}{8}};

	%Level1
	\draw pic {firstInnerNode={15}{28}{}{}{1}{1}{4}};
	\draw pic {innerNodeVar={50}{79}{}{}{1}{2}{4}};

	%Level2
	\draw pic {firstInnerNode={3}{14}{}{}{2}{3}{1}};
	\draw pic {innerNodeVar={32}{35}{}{}{2}{5}{0.8}};
	\draw pic {innerNodeNarrow={39}{43}{46}{}{2}{6}};
	\draw pic {innerNodeVar={84}{89}{92}{}{2}{8}{0.8}};

	%Level2.5
	\draw pic {firstInnerNode={19}{26}{}{}{2.5}{4}{4}};
	\draw pic {innerNodeVar={65}{69}{71}{}{2.5}{7}{4}};

	%Verbindungspfeile
	\draw pic {connect={0}{0}{1}};
	\draw pic {connect={0}{1}{2}};
	\draw pic {connect={1}{0}{3}};
	\draw pic {connect={1}{1}{4}};
	\draw pic {connect={1}{2}{5}};
	\draw pic {connect={2}{0}{6}};
	\draw pic {connect={2}{1}{7}};
	\draw pic {connect={2}{2}{8}};
	\end{tikzpicture}
\end{center}
\end{note}
