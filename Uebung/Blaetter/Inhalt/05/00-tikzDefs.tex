
\usetikzlibrary{arrows, chains, positioning}

\tikzset{
defaultNode1/.style={rectangle, draw, minimum height=8mm, minimum width=2mm, node distance=0mm},
defaultNode2/.style={defaultNode1, minimum width=0mm},
greyNode/.style={fill=lightgray},
secondNode/.style={on chain},
}

%Erster Innerer Knoten einer Ebene.
%\param #1 : 1. Referenzschlüsseleintrag
%\param #2 : 2. Referenzschlüsseleintrag
%\param #3 : 3. Referenzschlüsseleintrag
%\param #4 : 4. Referenzschlüsseleintrag
%\param #5 : Ebene auf der sich der Knoten befindet
%\param #6 : Name des Knotens (am besten fortlaufende Zahlen verwenden)
%\param #7 : Offset in x-Richtung
\tikzset{pics/firstInnerNode/.style n args={7}{
code={
	\node[defaultNode, greyNode](#6+0) at (#7,-#5*2) {};
	\chainin (#6+0);
	\node[defaultNode, secondNode]{#1};
	\node[defaultNode, secondNode, greyNode](#6+1){};
	\node[defaultNode, secondNode]{#2};
	\node[defaultNode, secondNode, greyNode](#6+2){};
	\node[defaultNode, secondNode]{#3};
	\node[defaultNode, secondNode, greyNode](#6+3){};
	\node[defaultNode, secondNode]{#4};
	\node[defaultNode, secondNode, greyNode](#6+4){};
}}}

%Innerer Knoten.
%\param #1 : 1. Referenzschlüsseleintrag
%\param #2 : 2. Referenzschlüsseleintrag
%\param #3 : 3. Referenzschlüsseleintrag
%\param #4 : 4. Referenzschlüsseleintrag
%\param #5 : Ebene auf der sich der Knoten befindet
%\param #6 : Name des Knotens (am besten fortlaufende Zahlen verwenden)
\tikzset{pics/innerNode/.style n args={6}{
code={

	\node[defaultNode,node distance=12mm, secondNode, greyNode](#6+0){};
	\node[defaultNode, secondNode]{#1};
	\node[defaultNode, secondNode, greyNode](#6+1){};
	\node[defaultNode, secondNode]{#2};
	\node[defaultNode, secondNode, greyNode](#6+2){};
	\node[defaultNode, secondNode]{#3};
	\node[defaultNode, secondNode, greyNode](#6+3){};
	\node[defaultNode, secondNode]{#4};
	\node[defaultNode, secondNode, greyNode](#6+4){};
}}}

%Innerer Knoten, variable Distanz.
%\param #1 : 1. Referenzschlüsseleintrag
%\param #2 : 2. Referenzschlüsseleintrag
%\param #3 : 3. Referenzschlüsseleintrag
%\param #4 : 4. Referenzschlüsseleintrag
%\param #5 : Ebene auf der sich der Knoten befindet
%\param #6 : Name des Knotens (am besten fortlaufende Zahlen verwenden)
\tikzset{pics/innerNodeVar/.style n args={7}{
code={
	\node[defaultNode,node distance=#7, secondNode, greyNode](#6+0){};
	\node[defaultNode, secondNode]{#1};
	\node[defaultNode, secondNode, greyNode](#6+1){};
	\node[defaultNode, secondNode]{#2};
	\node[defaultNode, secondNode, greyNode](#6+2){};
	\node[defaultNode, secondNode]{#3};
	\node[defaultNode, secondNode, greyNode](#6+3){};
	\node[defaultNode, secondNode]{#4};
	\node[defaultNode, secondNode, greyNode](#6+4){};
}}}

%Innerer Knoten, weniger Abstand.
%\param #1 : 1. Referenzschlüsseleintrag
%\param #2 : 2. Referenzschlüsseleintrag
%\param #3 : 3. Referenzschlüsseleintrag
%\param #4 : 4. Referenzschlüsseleintrag
%\param #5 : Ebene auf der sich der Knoten befindet
%\param #6 : Name des Knotens (am besten fortlaufende Zahlen verwenden)
\tikzset{pics/innerNodeNarrow/.style n args={6}{
code={
	\node[defaultNode,node distance=3mm, secondNode, greyNode](#6+0){};
	\node[defaultNode, secondNode]{#1};
	\node[defaultNode, secondNode, greyNode](#6+1){};
	\node[defaultNode, secondNode]{#2};
	\node[defaultNode, secondNode, greyNode](#6+2){};
	\node[defaultNode, secondNode]{#3};
	\node[defaultNode, secondNode, greyNode](#6+3){};
	\node[defaultNode, secondNode]{#4};
	\node[defaultNode, secondNode, greyNode](#6+4){};
}}}

%Erster Blattknoten einer Ebene.
%\param #1 : 1. Referenzschlüsseleintrag
%\param #2 : 1. Nutzdateneintrag
%\param #3 : 2. Referenzschlüsseleintrag
%\param #4 : 2. Nutzdateneintrag
%\param #5 : Ebene auf der sich der Knoten befindet
%\param #6 : Name des Knotens (am besten fortlaufende Zahlen verwenden)
%\param #7 : Offset in x-Richtung
\tikzset{pics/firstLeafNode/.style n args={7}{
code={
	\node[defaultNode](#6+0) at (#7,-#5*2) {#1};
	\chainin (#6+0);
	\node[defaultNode, secondNode]{#2};
	\node[defaultNode, secondNode]{#3};
	\node[defaultNode, secondNode]{#4};
}}}

%Blattknoten.
%\param #1 : 1. Referenzschlüsseleintrag
%\param #2 : 1. Nutzdateneintrag
%\param #3 : 2. Referenzschlüsseleintrag
%\param #4 : 2. Nutzdateneintrag
%\param #5 : Ebene auf der sich der Knoten befindet
%\param #6 : Name des Knotens (am besten fortlaufende Zahlen verwenden)
\tikzset{pics/leafNode/.style n args={6}{
code={
	\node[defaultNode,node distance=3.5mm, secondNode](#6+0){#1};
	\node[defaultNode, secondNode]{#2};
	\node[defaultNode, secondNode]{#3};
	\node[defaultNode, secondNode]{#4};
}}}

%Verbindungspfeil zeichnen
%\param #1 : Name des Vaterblocks
%\param #2 : Index des Pointers (beginnend bei 0)
%\param #3 : Name des Kindblocks
\tikzset{pics/connect/.style n args={3}{
code={
	\draw[->] (#1+#2.south) -- (#3+0.north);
}}}
