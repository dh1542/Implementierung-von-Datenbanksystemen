Willkommen zu den Übungen zu Implementierung von Datenbanksystemen!
\begin{normalText}
Die Übungsblätter enthalten sowohl Präsenzaufgaben, die in den Übungen erarbeitet und besprochen werden, als auch Vertiefungsaufgaben.
Zu den Präsenzaufgaben stellen wir einige Zeit nach der Behandlung in den Übungen ausführliche Musterlösungen über StudOn zur Verfügung.
Vertiefungsaufgaben sind mit einem oder zwei Sternen (* bzw. **) am linken Rand gekennzeichnet und dienen dem selbständigen Üben.
Zu den Vertiefungsaufgaben gibt es absichtlich keine Lösungen, um dazu anzuregen, diese auch wirklich selbst zu bearbeiten.
Fragen zu diesen Aufgaben können Sie aber gerne in der Intensivierungsübung oder im StudOn-Forum stellen und dort auch Ihre Lösungen diskutieren.
Dort bem\"uhen wir uns auch um zeitnahe Anmerkungen von unserer Seite zu offenen Fragen, nicht jedoch im FSI-Forum.
Weiterhin steht Ihnen für die Aufgaben mit einem Stern (*) während des in StudOn beim jeweiligen Übungsblatt angegebenen Zeitraums auch die Möglichkeit offen, freiwillig Lösungen über StudOn einzureichen und korrigieren zu lassen.
Für die Aufgaben mit zwei Sternen (**) stehen Ihnen Selbsttests zur Verfügung, mit denen Sie nach Ablauf der Übungswoche selbstständig Ihre Lösungen überprüfen können.
\end{normalText}
\begin{beamerText}
Übungsblätter bestehen aus Präsenz- und Vertiefungsaufgaben.
Musterlösungen für Präsenzaufgaben werden auf StudOn veröffentlicht.
Vertiefungsaufgaben können freiwillig auf StudOn abgegeben werden und werden korrigiert.
Alternativ gibt es zu einigen Vertiefungsaufgaben auch Selbsttests.
Diese finden Sie auch auf StudOn.
Bei Fragen zu den Aufgaben stehen Ihnen die Intensivierungsübung und das StudOn-Forum zur Verfügung.
\end{beamerText}
