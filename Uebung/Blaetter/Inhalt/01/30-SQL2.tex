\section{SQL 2}
Gegeben ist der folgende Teil eines einfachen Datenbankschemas:

\texttt{Professoren(\underline{PersNr}, Name, Rang, Raum)}\\
\texttt{Studenten(\underline{MatrNr}, Name, Semester)}\\
\texttt{Vorlesungen(\underline{Vorlnr}, Titel, SWS, gelesenVon\normaltxt{[Professoren]})}\\
\texttt{voraussetzen(\underline{Vorgaenger\normaltxt{[Vorlesungen]}, Nachfolger\normaltxt{[Vorlesungen]}})}\\
\texttt{hoeren(\underline{MatrNr\normaltxt{[Studenten]}, VorlNr\normaltxt{[Vorlesungen]}})}\\
\texttt{Assistenten(\underline{PersNr}, Name, Fachgebiet, Boss\normaltxt{[Professoren]})}\\
\texttt{pruefen(\underline{MatrNr\normaltxt{[Studenten]}, VorlNr\normaltxt{[Vorlesungen]}}, PersNr\normaltxt{[Professoren]}, Note)}
\cprotEnv
\begin{normalText}

Sie können Ihre Ergebnisse gerne online unter \href{https://hyper-db.de/interface.html}{hyper-db.de/interface.html} überprüfen.

\begin{enumerate}[a)]
	\item Erstellen Sie eine SQL-Anfrage, die alle Studierenden nach ihrer 
	bisherigen Studiendauer sortiert, beginnend mit der längsten bisherigen 
	Studiendauer.
	Ausgegeben werden sollen nur der Name und die Matrikelnummer.

\cprotEnv
	\begin{note}
	\begin{lstlisting}
SELECT Name, MatrNr
FROM   Students
ORDER BY Semester DESC;
	\end{lstlisting}
	\end{note}

	\item Erstellen Sie eine SQL-Anfrage, die alle Professoren mit Namen und der Anzahl der von ihnen jeweils gehaltenen Vorlesungen ausgibt.
	Die Ausgabe soll absteigend nach Anzahl der Vorlesungen sortiert sein.

\cprotEnv
	\begin{note}
	\begin{lstlisting}
SELECT Professoren.Name, count(*) AS anzahl_der_vorlesungen
FROM   Professoren, Vorlesungen
WHERE  PersNr = gelesenVon
GROUP BY Name, PersNr
ORDER BY anzahl_der_vorlesungen DESC;
	\end{lstlisting}
	\end{note}

	\item Erstellen Sie eine SQL-Anfrage, die alle Professoren mit mehr als einer Vorlesung alphabetisch nach Namen sortiert auflistet.
	Neben dem Namen der Professoren und der Anzahl der Vorlesungen soll jeweils auch die Raumnummer der Professoren, sowie die Rang der Professoren ausgegeben werden.

	Stellen Sie die SQL-Anfrage dann so um, dass die Sortierung erst absteigend nach der Anzahl der Vorlesung, dann alphabetisch nach Namen erfolgt.

\cprotEnv
	\begin{note}
	\begin{lstlisting}
SELECT Name, Raum, Rang, count(*) as anzahlVorlesungen
FROM   Professoren, Vorlesungen
WHERE  Professoren.PersNr = Vorlesungen.gelesenVon
GROUP BY Professoren.PersNr, Name, Raum, Rang
HAVING count(*) >= 2
ORDER BY Name ASC;
	\end{lstlisting}

	Da ASC die Standard-Reihenfolge ist, wenn nicht explizit DESC angegeben wird, kann es auch weggelassen werden.

	\begin{lstlisting}
SELECT Name, Raum, Rang, count(*) as anzahlVorlesungen
FROM   Professoren, Vorlesungen
WHERE  Professoren.PersNr = Vorlesungen.gelesenVon
GROUP BY Professoren.PersNr, Name, Raum, Rang
HAVING count(*) >= 2
ORDER BY anzahlVorlesungen DESC, Name ASC;
	\end{lstlisting}
	\end{note}


	\item Erstellen Sie eine SQL-Anfrage, die die Anzahl der Professoren, welche mehr als eine Vorlesung anbieten, sowie die Gesamtanzahl der von diesen Professoren angebotenen Vorlesungen angibt.

	Dazu benötigen Sie das Ergebnis der vorherigen Aufgabe.
	Erstellen Sie eine View, die dieses Ergebnis zur Verfügung stellt.

	Diskutieren Sie Vor- und Nachteile der Verwendung einer View.
	Überlegen Sie sich, wie man das View-Konzept erweitern könnte, so dass es sich zur Beschleunigung der Anfrageausführung eignet (die Lösung lernen Sie noch im weiteren Verlauf von IDB kennen).
	
	\textit{Hinweis:} Diese Teilaufgabe lässt sich leider nicht online überprüfen, da dort Views nicht erlaubt sind. Alternativ können sie ein With-Statement verwenden (Vgl. TPC-H Query 15).

\cprotEnv
	\begin{note}
	\begin{lstlisting}
CREATE VIEW bigProfs AS
(
SELECT count(*) as anzahlVorlesungen
FROM   Vorlesungen
GROUP BY gelesenVon
HAVING count(*) >= 2
);
	\end{lstlisting}

	Diskussion View:

	Ein View wird jedes Mal, wenn er verwendet wird, neu ausgewertet.
	Somit bleibt ein View auch aktuell, wenn sich die zugrundeliegenden Daten ändern.
	Dies kann allerdings zu einer schlechten Performanz bei komlexeren Anfragen führen.

	Mögliche Alternativen sind Materialized Views (werden später in IDB behandelt) oder Stored Procedures.
	Bei dieser einmaligen Anfrage sollte normalerweise keine View erstellt werden, sondern stattdessen die With-Klausel verwendet werden.
	Diese ist den Teilnehmern aber nicht aus KonzMod bekannt.

	\begin{lstlisting}
SELECT count(*) AS professoren,
       sum(anzahlVorlesungen) AS totaleAnzahlVorlesungen
FROM   bigProfs;
	\end{lstlisting}

	Da wir die View laut Aufgabenstellung nicht mehr weiter verwenden wollen:

	\begin{lstlisting}
DROP VIEW bigProfs;
	\end{lstlisting}
	\end{note}


%	\item Erstellen \deepen Sie eine SQL-Anfrage, die alle Braumeister mit Brauer-ID, Name und Geburtsdatum ausgibt, welche mindestens ein Bier mit mehr als 6\% Vol. kreiert haben.

%	\begin{note}
%	\begin{lstlisting}
%SELECT *
%FROM   brewmasters
%WHERE EXISTS
%(SELECT *
% FROM   beers
% WHERE  beers.brewmaster = brewmasters.brewer_id
% AND    alcohol_by_volume > 6);
%	\end{lstlisting}
%	\end{note}
% TODO Das geht einfacher mit einem Join. Durch eine Aufgabe ersetzen, bei der exists sinnvoller eingesetzt wird.
%
%	\item Erstellen \deepen Sie eine SQL-Anfrage, die alle Braumeister mit Brauer-ID, Name und Geburtsdatum ausgibt, welche noch keine fünf Biere kreiert.
%
%	\begin{note}
%	\begin{lstlisting}
%SELECT brewmasters.brewer_id,
%       brewmasters.name,
%       brewmasters.date_of_birth
%FROM   brewmasters, beers
%WHERE  beers.brewmaster = brewmasters.brewer_id
%GROUP BY brewmasters.brewer_id,
%         brewmasters.name,
%         brewmasters.date_of_birth
%HAVING count(*) < 5;
%	\end{lstlisting}
%	\end{note}
% TODO Stattdessen eine Aufgabe ergänzen, die einen Mengenoperator verwendet

	\item Erstellen Sie ein SQL-Statement, mit der Sie sich selbst, mit 
	Schreibfehler im Namen, in die Liste der Studierenden hinzufügen.

	Korrigieren Sie anschließend den Schreibfehler in der Datenbank.

	\textit{Hinweis:} Diese Teilaufgabe lässt sich leider nicht online überprüfen, da Änderungsoperationen nicht erlaubt sind.

\cprotEnv
	\begin{note}
	\begin{lstlisting}
INSERT INTO studenten
VALUES (42, 'Max Mutterstudent', 5);
	\end{lstlisting}

	\begin{lstlisting}
UPDATE studenten
SET   name = 'Max Musterstudent';
WHERE MatrNr = 42
	\end{lstlisting}
	\end{note}

\end{enumerate}
\end{normalText}
