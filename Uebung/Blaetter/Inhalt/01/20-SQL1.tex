\section{SQL 1}
Gegeben ist der folgende Teil eines einfachen Datenbankschemas:

\texttt{songs(\underline{title, artist}, length, album[albums])}

\texttt{producers(\underline{name}, date\_of\_birth)}

\texttt{albums(\underline{title}, producer[producers], release\_year, genre)}

\begin{enumerate}[a)]
	\item Erstellen Sie eine SQL-Anfrage, die alle Titel von Ray Charles' Songs liefert, die höchstens 180 Sekunden dauern!

\cprotEnv
	\begin{solution}
	\begin{lstlisting}
SELECT title
FROM   songs
WHERE  artist = 'Ray Charles' and length <= 180;
	\end{lstlisting}
	\end{solution}


	\item Erstellen Sie eine SQL-Anfrage, die jeden Songtitel zusammen mit dem Geburtsdatum des Produzenten des Albums ausgibt, auf dem der Song erschienen ist.

\cprotEnv
	\begin{solution}
	\begin{lstlisting}
SELECT s.title, p.date_of_birth
FROM   songs s, producers p, albums a
WHERE  s.album = a.title AND a.producer = p.name;
	\end{lstlisting}
	\end{solution}


	\item Erstellen Sie eine SQL-Anfrage, die eine aufsteigend nach Künstlername sortierte Liste der Künstler und der Anzahl an Songs, die der jeweilige Künstler in der Datenbank hat, zurückgibt.
	Berücksichtigen Sie dabei nur Songs, die auf einem Album erschienen sind und berücksichtigen Sie nur Künstler, zu denen es mindestens 5 solcher Songs gibt.

\cprotEnv
	\begin{solution}
	\begin{lstlisting}
SELECT   artist, count(album) AS anzahl
FROM     songs
GROUP BY artist
HAVING   count(album) >= 5
ORDER BY artist ASC;
	\end{lstlisting}

	\begin{note}
	Durch \texttt{count(album)} kann auf eine Prüfung von \texttt{album IS NOT NULL} verzichtet werden.
	\end{note}

	Zusatzfrage: Wie könnte man das gleiche Ergebnis ohne Verwendung von \texttt{GROUP BY} und \texttt{HAVING} erzielen?
	\end{solution}

\cprotEnv
	\begin{note}
	Mit korrelierten Unteranfragen:
	\begin{lstlisting}
SELECT  DISTINCT artist,
        (SELECT count(album)
         FROM songs si
         WHERE si.artist = so.artist)
        AS anzahl
FROM    songs so
WHERE   (SELECT count(album)
         FROM songs si
         WHERE si.artist = so.artist)
        >= 5
ORDER BY artist ASC
    \end{lstlisting}

	\end{note}


	\item Erstellen Sie eine SQL-Anfrage, die die 5 Alben mit der durchschnittlich längsten Song-Dauer in einem Ranking ausgibt, wobei das Album mit den durchschnittlich längsten Songs Rang 1 belegt.

\cprotEnv
	\begin{solution}

	Als Top-N-Anfrage mit Join, wie sie aus KonzMod bekannt ist:
	\begin{lstlisting}
WITH avg_songlength AS
(
  SELECT   album, AVG(length) AS avg_song_length
  FROM     songs
  WHERE album IS NOT NULL
  GROUP BY album
)
SELECT   COUNT(*) AS rang, a1.album, a1.avg_song_length
FROM     avg_songlength a1, avg_songlength a2
WHERE    a1.avg_song_length <= a2.avg_song_length
GROUP BY a1.album, a1.avg_song_length
HAVING   COUNT(*) <= 5
ORDER BY rang ASC;
	\end{lstlisting}

	Unter Verwendung von limited fetch aus SQL:2008/SQL:2011 ohne Ausgabe des Rangs:

	\begin{lstlisting}
SELECT   album, AVG(length) AS avg_song_length
FROM     songs
WHERE album IS NOT NULL
GROUP BY album
ORDER BY avg_song_length DESC
FETCH FIRST 5 ROWS ONLY;
	\end{lstlisting}

	Zusatzfragen: Was passiert, wenn zwei Alben die gleiche durchschnittliche Songdauer haben? Was passiert mit Songs, die auf keinem Album erschienen sind?
	\end{solution}

\cprotEnv
	\begin{note}
	Unter Verwendung von analytic functions mit Ausgabe des Rangs.
	Nur als Ergänzung, muss in den normalen Übungen nicht besprochen werden.

	\begin{lstlisting}
SELECT * FROM (
  SELECT
    RANK() OVER(ORDER BY AVG(length) DESC) AS rang,
    album,
    AVG(length) AS avg_song_length
  FROM songs
  WHERE album IS NOT NULL
  GROUP BY album
)
WHERE rang <= 5
ORDER BY rang ASC;
	\end{lstlisting}

	Die Rangvergabe bei mehrfach vorhandenen durchschnittlichen Songdauern ist hier anders als bei der Top-N-Anfrage mit Joins.
	Bei der Join-Anfrage fallen Ränge kleiner des mehrfach vergebenen Rangs weg, bei Verwendung von \texttt{RANK} Ränge größer des mehrfach vergebenen Rangs.

	\end{note}


	\item Erstellen Sie eine SQL-Anfrage, die alle Songs ausgibt, die länger sind als der Durchschnitt aller Songs.
	Erweitern Sie die Anfrage dann so, dass sie alle Songs ausgibt, die länger sind als der Durchschnitt aller Songs des jeweiligen Albums.

\cprotEnv
	\begin{solution}
	\begin{lstlisting}
SELECT *
FROM   songs s1
WHERE  s1.length >
(
  SELECT avg (s2.length)
  FROM songs s2
);

SELECT *
FROM   songs s1
WHERE  s1.length >
(
  SELECT avg (s2.length)
  FROM songs s2
  WHERE s1.album = s2.album
);
	\end{lstlisting}
Bei der erweiterten Lösung handelt es sich um eine korrelierte Unteranfrage.
Dies bedeutet, dass Attribute der äußeren SQL-Anfrage und der Unteranfrage in der \texttt{WHERE}-Klausel miteinander verknüpft sind.
Dabei kann die Unteranfrage nicht ohne die äußere Anfrage durchgeführt werden.
Für jeden Satz der äußeren Anfrage wird die Unteranfrage einmal durchgeführt. (Optimierungen sind aber teilweise möglich)
	\end{solution}

	\item Erstellen \deepen Sie eine SQL-Anfrage, die alle Künstler liefert, die mit einem Song auf irgendeinem Album vertreten sind.
	Achten Sie darauf, dass jeder Künstler nur einmal gelistet wird, also keine Duplikate auftreten.

\cprotEnv
	\begin{note}
	\begin{lstlisting}
SELECT DISTINCT artist
FROM   songs
WHERE  album IS NOT NULL;
	\end{lstlisting}
	\end{note}

	\item \label{kuenstler_gesamtzahl} Erstellen \deepen Sie eine SQL-Abfrage, die zu jedem Künstler die Gesamtzahl seiner Songs listet.

\cprotEnv
	\begin{note}
	\begin{lstlisting}
SELECT   artist, count(*)
FROM     songs
GROUP BY artist;
	\end{lstlisting}
	\end{note}


	\item Erweitern \deepen Sie die Abfrage aus Aufgabe \ref{kuenstler_gesamtzahl}) so, dass die Ausgabe nach Anzahl veröffentlichter Songs absteigend sortiert wird.

\cprotEnv
	\begin{note}
	\begin{lstlisting}
SELECT   artist, count(*) AS veroeffentlichte_songs
FROM     songs
GROUP BY artist
ORDER BY veroeffentlichte_songs DESC;
	\end{lstlisting}
	\end{note}

\end{enumerate}
